\documentclass[conference]{IEEEtran}
\IEEEoverridecommandlockouts
% The preceding line is only needed to identify funding in the first footnote. If that is unneeded, please comment it out.
\usepackage{cite}
\usepackage{amsmath,amssymb,amsfonts}
\usepackage{algorithmic}
\usepackage{graphicx}
\usepackage{textcomp}
\usepackage{xcolor}
\usepackage{url}
\usepackage[linguistics]{forest}
\def\BibTeX{{\rm B\kern-.05em{\sc i\kern-.025em b}\kern-.08em
    T\kern-.1667em\lower.7ex\hbox{E}\kern-.125emX}}
\begin{document}

\title{Position Paper: Edge Clouds Multiple Control Planes Data Replication
Challenges\\
}

\author{\IEEEauthorblockN{Bohdan Dobrelia}
\IEEEauthorblockA{\textit{OpenStack platform} \\
\textit{Red Hat}\\
Poznan, Poland \\
bdobreli@redhat.com}
}

\maketitle

\begin{abstract}
Fog computing is an emerging paradigm aiming at bringing cloud functions closer
to the end users and data sources. Its control plane and deployments data
synchronization is a major challenge. Autonomy requirements expect even the
most distant edge sites always manageable, available for monitoring and
alerting, scaling up/down, upgrading and applying security fixes. Whenever
temporary disconnected sites are managed locally or centrally, some changes
and data need to be eventually synchronized back to the central site(s) with
having its merge-conflicts resolved for the central data hub(s). While some
data needs to be pushed from the central site(s) to the Edge, which might
require resolving data collisions at the remote sites as well. In this paper,
we position the outstanding data synchronization problems for OpenStack
platform becoming a cloud solution number one for fog computing. We define
the inter-cloud operational invariants based on that Always Available
autonomy requirement. We show that a causal consistent data replication is
the best match for the outlined operational invariants and there is a great
opportunity for designing such a solution for Edge clouds. Finally, the paper
brings vision of unified tooling to solve outstanding state synchronization
problems the same way for infrastructure owners, cloud operators and tenants
running stateful workloads hosted on OpenStack IaaS or OpenShift/Kubernetes
PaaS deployed in Edge clouds as multi-cloud workloads abstraction and
unification layer, to make it truly cloud-vendors agnostic and portable.
\end{abstract}

\begin{IEEEkeywords}
Open source software, Edge computing, Distributed computing, System
availability, Design
\end{IEEEkeywords}

\section{Introduction}
OpenStack is an Infrastructure-as-a-Service platform number one for private
cloud computing, and it becomes being so for fog computing as well.
Hybridization and multi-cloud trends for private clouds interconnected with
public clouds and Platform-as-a-Service (PaaS) solutions, like
OpenShift/Kubernetes, allow the containerization of micro-services oriented
workloads to emerge in a highly portable, self-contained and the hosting
cloud-agnostic way. Giving it massively distributed scale of fog computing and
bringing the data it operates closer to end users, opens great opportunities
for Internet of Things (IoT) and 5G telecommunication technologies,
which first of all requires low-latency and highly responsive interfaces always
available for end users.

Speaking of an always available, back to the system administration realities,
the Edge clouds control and management plane capabilities in such a massively
distributed world shall not fall behind as well. In Fog environments,
expectations for those capabilities are very different to the traditional cloud
environments. There you have a highly available control plane, and a management
layer for your cloud. From the geographical standpoint, it is also a mostly
still layout. Control/management nodes are hosted in well suited datacenters
and communicate over low-latency reliable connections, where network partitions
are considered as disasters. While in Fog clouds, being offline is rather a
normal state of things, where state can be synchronized eventually. Think of
autonomous unmanned drones controlled via compact mobile datacenters that
cannot always maintain its upstream links up. In 5G networks, smart mobile
applications, like speech or video recognition, go far beyond the computing
powers of tiny devices running it. So data have to travel with its consumers to
be always at hand and accessed via a low-latency connection from a nearest
compact datacenter. It has to be compact as maintaining thousands of classic
datacenters won't fit neither the most generous business model nor if limited
space/weight et al requirements. And such autonomy imposes multiple control and
configuration management planes to maintain local operations and manageability
of systems while being offline. In turn, that brings outstanding problems of
data replication and consistency.

\section{Background Concepts}
Aside of the established terms\cite{b3}, we define a few more for the data
processing and operational aspects:

\textbf{Deployment Data}: data that represents the configuration of
\textit{cloudlets}\cite{b3}, like API endpoints URI, or numbers of deployed
\textit{edge nodes}\cite{b3} in \textit{edge clouds}\cite{b3}. That data may
represent either the most recent state of a deployment, or arbitrary data
chunks/operations required to alter that state. When there is unresolved data
merging conflicts or queued operations pending for future execution, the most
recent state becomes the \textit{best known} one.
\begin{figure}[htbp]
\caption{Example Replication Topology}.
\begin{forest}
  [$\mathrm{EAL1}$
    [\textit{access edge layer}\cite{b3}
     [\textit{infrastructure edge}\cite{b3}
       [\textit{device edge}\cite{b3}]
     ]
    ]
    [$\mathrm{EAL2}$
      [$\mathrm{EAL3}$
        [$\mathrm{cloudlet1}$
        ]
      ]
      [$\mathrm{EAL4}$
        [$\mathrm{cloudlet2}$
          [$\mathrm{EAL5}$
            [$\mathrm{cloudlet3}$
          ]
          [$\mathrm{cloudlet4}$
          ]
        ]
      ]
    ]
  ]
]
\label{fig}
\end{forest}
\end{figure}

\textbf{Cloud Data}: similarly to deployment data, represents the most recent
or the best known internal and publicly visible state of cloudlets, like cloud
users or virtual routers. Cloud data also includes logs, performance and usage
statistics, state of message queues and the contents of databases. It may as
well represent either data chunks or operations targeted for some state
$\mathrm{S}$ transitioning into a new state $\mathrm{S'}$.

\textbf{Control Plane}: corresponds to any operations performed via cloudlets
API endpoints or command line tooling. E.g. starting a virtual machine
instance or a Kubernetes pod, or creating an OpenStack Keystone user. Such
operations are typically initiated by cloud applications, tenants or operators.

\textbf{Replication Topology}: represents a hierarchy for allowed state
replication flows and targets for operations on interconnected cloudlets,
including \textit{edge aggregation layers}\cite{b3}.
So effectively there is only a limitation for horizontally interconnected
cloudlets cannot be replicating its state nor targeting operations to each other.
This corresponds to the acyclic graph (tree) topology.

For the example graph (Fig.~\ref{fig}), the $\mathrm{infrastructure\ edge}$ can
replicate data, like contributing its local stats into the global view of
hardware and performance metrics, to the the edge aggregation layer
$\mathrm{EAL1}$ omitting the $\mathrm{access\ edge\ layer}$. Meanwhile the
latter can be pushing something, like deployment data changes, to the
$\mathrm{device\ edge}$ and $\mathrm{infrastructure\ edge}$. On the right side
of the graph, $\mathrm{cloudlet3}$ and $\mathrm{cloudlet4}$ cannot replicate to
each other, but can do to the edge aggregation layer $\mathrm{EAL1}$ either
directly or consequently via $\mathrm{EAL5}$, $\mathrm{cloudlet2}$,
$\mathrm{EAL4}$ and $\mathrm{EAL2}$, which is totally the replication
implementation specific. We also say that $\mathrm{EAL2}$ and $\mathrm{EAL1}$
are hierarchically upper associated with $\mathrm{EAL4}$ and $\mathrm{EAL5}$,
or just $\mathrm{EAL1}$ is an \textit{upper layer} for $\mathrm{EAL2}$, when we
mean that the latter is a potential subordinate of the former. Finally, we can
say that $\mathrm{EAL4}$ controls/manages $\mathrm{cloudlet2}$ and anything
sitting down of it.

That is, a control/manage-subordinate hierarchical association only defines
influence domains and does not imply state replication as a mandatory thing nor
imposes bidirectional data synchronization, but rather provides an opportunity
of one-way or bidirectional state replications and regulates a possibility of
issuing control/management operations. State replication may be performed from
subordinates to its upper associates, think of sending reports. Or vice versa,
think of sending directives to subordinates. While control/management
operations may only flow up-down, think of planning future work for
subordinates.

\textbf{Management Plane}: corresponds to administrative actions performed via
configuration and lifecycle management systems. Such operations are typically
targeted for cloudlets, like edge nodes, \textit{edge datacenters}\cite{b3},
or edge clouds. E.g. upgrading or reconfiguring cloudlets in a \textit{virtual
datacenter}\cite{b3}, or scaling up edge nodes. And typically initiated by
cloud infrastructure owners. For some cases, like Baremetal-as-a-Service,
tenants may as well initiate actions executed via the management plane.
Collecting logs, performance and usage statistics for monitoring and alerting
systems also represents the management plane operations, although it operates
with the cloud data.

When we refer to an edge aggregation layer and cloudlets under its
control/management, we mean exactly any operations executed via the
control/management planes of that edge aggregation layer and targeted for
cloudlets sitting down the nested connections graph. And a replication topology
regulates allowed targets for operations and state replication.

\textbf{Data Replication Conflict}: according to\cite{b1}, two operations on
the same target are in conflict if they are not related by causality.

\textbf{Always Available}: the operational mode of the control and management
planes that corresponds to \textit{sticky available causal
consistency}\cite{b4} data replication models, i.e. RTC (\textit{Real-Time
Causal}\cite{b2}), or \textit{causal+}\cite{b1}. Depending on the consistency
model choices, there may be additional constraints:
\begin{itemize}
  \item the stickiness property is a mandatory constraint for sticky available
    causal consistent systems. That is: ``on every non-faulty node, so long
    as clients only talk to the same servers, instead of switching to new
    ones''\cite{b4}.
  \item the real-time constraint is keeping the system time synchronized for
    all cloudlets. That is a mandatory constraint for RTC.
  \item \textit{one-way convergence}\cite{b2} is a mandatory for RTC.
\end{itemize}
Causal+ and RTC consistency ensure ordering of relative operations, i.e. all
causal related writes can be seen in the same order by all processes (connected
to the same server). It is also known that ``the causal consistency supports
non-blocking operations; i.e. processes may complete read or write operations
without waiting for global computation. Therefore, the causal consistency
overcomes the primary limit of stronger criteria: communication
latency''\cite{b6}. All that provides the best causal consistency guarantees we
can get for today for always available systems.

\section{Analysis and Discussion}
\subsection{Autonomy Requirements}
We define always available autonomy for cloudlets as the following strict
requirements:

\subsubsection{Maintain multiple control/management planes}
Autonomous cloudlets may only operate as always available when having multiple
control/management planes. For example, when an offline cloudlet cannot start
virtual machines or containers, that violates the autonomy requirements.

\subsubsection{Provide no read-only or blocking access for inter-cloudlet
operations}
Any operations performed on cloudlets state, despite its aliveness/failure
conditions as it's shown in the global view of upper edge aggregation layers,
fit consistency models that allow the involved control/management planes
operating as always available, and there is no limitations, like read-only or
blocking access. Operations may be queued for future processing in order to
meet this autonomy requirement though. Internal state of cloudlets is allowed
to keep its failure modes unchanged. E.g. for a local or distributed
Galera/MariaDB database or a Etcd cluster, its standard quorum requirements
apply for internal data transactions. That is a transitioning requirement until
the internal cloudlets state can be migrated to causal consistent data storages
as well.

\subsubsection{Provide a local control/management plane, whenever possible}
Operations on cloudlets can be scheduled at any given moment of time. For
offline/partitioned autonomous cloudlets, that can be done via local
control/management plane, if it exists and not failed. The same transitioning
considerations for internal cloudlets state apply.

\subsubsection{Support fully autonomous (offline) cloudlets}
Aggregation edge layer cloudlets should allow for arbitrary or all of its
managed/controlled cloudlets running fully autonomous long-time or indefinitely
long, queuing any operations targeted for such cloudlets. That is, to have
failure domains size of a 1. For a permanently disconnected cloudlet it may make
more sense though to detach it from its adjacent aggregation layer and/or
reorganize its place taken in the replication topology.

\subsubsection{Queue operations to keep it always available at best
effort}
Queued operations have to be eventually replied if/after the control/management
plane capabilities restored, or dropped (e.g. expired) otherwise. That poses a
lazy replication principle. If there is intermediate aggregation edge layers
down the way to the target of the queued operations, the queue may be shared
across each of the involved aggregation edge layers or optionally,
queued operations may be distributed across not shared queues. That should
reduce the associated memory and disc pressure for aggregation layers.

\subsubsection{Provide a global view for cloudlets aliveness state}
Global view of cloudlets needs to be periodically presented for at least one of
upper aggregation edge layers. For example, with
the state marks, like ``unknown'', or ``autonomous''; ``synchronizing'',
``connected''; ``failed'', or ``disconnected'', or ``fenced'', if and only if
it is confirmed as failed, or manually disconnected, or fenced automatically.
That poses the aliveness of the control/management planes principle.

\subsection{Operational Invariants}
To be always available as we defined it, control and management planes of
cloudlets should provide the following operational capabilities
(\textit{invariants} hereafter):

\subsubsection{Keep control planes always available at best effort}
CRUD (Create, Read, Update and Delete) operations on cloud data can always be
requested via API/CLI of local cloudlets or upper layers. The same queuing
requirements apply as it is defined for the autonomy requirements.

\subsubsection{Do not wait for edge aggregation layers control planes for local
operations}
Local CRUD operations for offline cloudlets, if its control plane exists and
not failed, can be processed without waiting for the upper aggregation layers
to recover its control over the cloudlets. When a cloudlet is running only
compute/storage resources, it cannot meet this requirement.

\subsubsection{Allow local scaling of infrastructure edge nodes without waiting
for management planes of upper aggregation layers}
Similarly the to control plane operations, deployed infrastructure edge nodes
can always be scheduled for scaling up/down by the cloudlets local management
planes, if it is possible (a cloudlet may be relying on the remote
configuration management only), or via its upper layers. Same queuing
requirements apply for operations.

\subsubsection{Allow hotfixes and kernel/software updates applied locally for
cloudlets}
Security patches and minor system updates, including kernel upgrades, can
always be scheduled for installation by the same meanings (via operations
issued locally or by the associated aggregation layers, including the same
queuing requirements).

\subsubsection{Allow major software versions upgrades applied locally for
cloudlets}
Similarly, major versions of system components, like OpenStack or
OpenShift/Kubernetes platforms, can be always scheduled for upgrades, using the
same meanings as above.

\subsubsection{Provide an extended global view for cloudlets}
Additionally to the aforementioned global view for cloudlets control/management
plane aliveness state marks, there needs to be a periodically updated global
view for each of the edge aggregation layers into its controlled/managed
cloudlets, at least the adjacent ones, for the key system administration
aspects, like hardware status, power management, systems state logging,
monitoring and alerting events, performance and metering statistics.

\subsection{State Replication Consistency Requirements}
As it follows from the defined always available autonomy requirements and
operational invariants, we define the following data replication
requirements\footnote{when we refer to just \textit{data} or \textit{state}, we
intentionally do not differentiate either that is deployment or cloud data, or
queued API/CLI operations, to be replicated/replayed for management or control
planes. That poses the \textbf{unified approach principle}}:

\subsubsection{Incorporate convergent conflict handling\cite{b1}}
Data replication conflicts can be resolved automatically, or by hand and
maintained as causal related. The conflicts resolving strategies and rules
should be customizable, like ``last writer wins'' or ``return them all''. After
the conflicts resolved, the data may be considered causal related, that is by
definition\cite{b1} of the data conflicts in eventually consistent systems.

\subsubsection{Prefer one-way convergence in replication topologies}
As far as the replication topology and queuing capabilities allow
that, causal related data can be replicated across cloudlets. Prefer one-way
convergence and avoid bidirectional replication whenever possible.

\subsubsection{Bidirectional replication is only a nice to have requirement}
Bidirectional (two-way convergence) data replication is not a strict
requirement but is nice to have. Indeed, some state needs to be replicated
one-way from aggregation edge layer to cloudlets under its control/management,
like virtual machine or hardware provisioning images data. While logs,
performance and metering statistics may be collected only from cloudlets to its
upper layers.

OpenStack/Kubernetes, have yet support for neither causal consistent storage
backends for its cloud/deployment data, nor client libraries that could drive
replication of casual related state. That poses an open opportunity for
developers and system architects to design and implement such state replication
tooling for multiple control/management planes.

OpenStack cloud data is normally stored in databases via transactions based on
stronger than causal \textit{unavailable}\cite{b4} data consistency models,
e.g. \textit{serializable}\cite{b4}, or \textit{repeatable read}\cite{b4}.
OpenShift/Kubernetes clusters state, some of SDN (Software Defined
Network) solutions are backed with Etcd, which also only supports
the stronger than causal consistency models. Those cannot tolerate high network
latency, serve two-way convergence only, therefore do not scale for potentially
dozens of data replicas.

From the other side, the weaker consistency models, which is \textit{total
available}\cite{b4}, provide a poor alternative that brings greatly increased
implementation complexity. E.g. \textit{monotonic atomic view}\cite{b4} would
require handling of \textit{fuzzy reads}\cite{b4}, \textit{phantoms}\cite{b4},
discarded write-only transactions, empty state returned for any reads.

\subsection{Data Replication Challenges}
All that brings us to challenges that need to be addressed for multiple
control/management planes:
\begin{itemize}
  \item categorizing control/management operations and data flows associated
    with it, then grouping those into particular replication topologies. Such
    groups may be identified by multiple metrics, like communication latency,
    network partition duration tolerance for offline cloudlets, one- or two-way
    convergence based, a shared causal data storage or client libraries
    implementation specific. For the latter case, either it should allow
    replicating data low-level, or replaying queued operations at API level.
    Finally, that grouping should also be done for built-in state replication
    capabilities of components. Depending on the constraints for latency or
    maximum nubmers of replicas, a distributed control/management plane may be
    a fit for some types of virtual data centers\cite{b3}.
  \item in the end, the final design should not bring excessive operational
    overhead, like if maintaining all of the identified replication topologies
    simultaneously for a deployment, and require not too much of human care,
    but still meet the unified approach principle as we defined it earlier.
  \item picking/combining identified replication topologies to use with each of
    the involved system components, like identity provider or images serving
    services. Ideally, (almost) stateless components will not require data
    replication aside of built-in capabilities of a single distributed control
    plane.
  \item designing strategies and rules for conflicts resolving based on picked
    replication topologies identified the previous steps. A ``last writer
    wins'' may be a good fit for database/KVS conflicting data synchronization,
    while ``return them all'' seems the best choice for manual or artificial
    intelligence driven/assisted ``smart'' conflict resolvers.
  \item keeping state replication topologies efficient, e.g. distributing
    locally queued operations targeted for offline cloudlets by nesting more of
    the edge aggregation layers underneath. Also, stateless or almost stateless
    components may be happy with only built-in capabilities of a single
    distributed control plane.
  \item abiding the unified approach/architecture principles for Edge clouds
    IaaS, optional Paas, and edge-native workloads as well.
\end{itemize}

\subsection{Vision of a Unified Deployment/Cloud State Replication Design}
The definition we made for always available distributed systems self-explains
why the causal consistent state replication is the best match for the
massively distributed cloudlets autonomy requirements and operational
invariants as we defined those.

The vision of the unified architecture for future state replication tooling
imposes it should be solving the multiple control/management planes data
synchronization problems for IaaS, PaaS and end users consuming it as
Replication-as-a-Service. Although generic version control systems, like Git,
might fit all cases for deployment data replicating and conflicts resolving,
that would break the unified design approach for cloud data/state replication.

Client libraries implementing causal consistent data replication and
customizable conflicts resolving rules may provide a unification layer for
different underlying databases/KVS (Key Value Storage). The replication will be
effectively acting as database/KVS-to-database/KVS data synchronization tooling
syncing data at a database level. The main benefit for such an approach is no a
shared data storage needed. Instead, the underlying local to cloudlets data
storages may keep operating as is, share nothing and provide unavailable
consistency models stronger than causal consistency. And cloudlets may keep
using different solutions for its local data storages as far as the state
replication tooling may support such backends.

Additionally, client libraries may replicate not only data but operations at an
API level\footnote{For OpenStack Nova, there had been an example for such an
API level replication, that is a Cells V1 protocol. But it had been deprecated
as real cells v1 deployments required constant human care and feeding
operationally} as well, i.e. resolve conflicting operations on-fly, then apply
the resulting causal related operations for its original targets, effectively
replicating changes at an API level. Operations queued by the
control/management planes, including those targeted for offline cloudlets, may
be also processed that way.

\section{Related Work}
COPS\cite{b1} formally proves implementation of a client library and highly scalable
tooling for causal+ data operations. By design, it does not impose any
real-time constraints and supports a single edge datacenter failure. The real
tooling made off that base, may be operating on top of the not shared local
cloudlets databases, or KVS, that provide the stronger consistency guarantees
by the costs of reduced availability for local services. That would work as
weaker consistency guarantees work well, when built on top of the stronger
ones, and provide an always available global view of cloudlets for upper
layers. Replicating the state changes via causal related operations and
conflicts resolving via custom handlers is that COPS covers as well.

Global causal consistent databases\cite{b6} describes alternative solutions, if
one can serve on the large scale and provide two-way convergence over high
latency networks. Or otherwise, multiple shared instances of it, each serving
to its dedicated virtual datacenter. A causal consistent data storage is the
most unified and also the most simple way of exposing it for edge-native
workloads as Replication-as-a-Service. The downside is such a thing has to be
supported as a control/management planes data backend for IaaS and PaaS itself.
So locally, cloudlets would have to be switched to use that data backend as
well.

RainbowFS\cite{b7} expands on the consistency models and building tools topics.
\begin{itemize}
  \item \textbf{Open questions}: does COPS work for multiple datacenters
    failure events, up to failure domains of a size of a 1? Is COPS applicable
    for two-way convergent systems, in terms of\cite{b2}, for bidirectional
    causal+ replications? Any open source implementations exist for RTC/COPS?
  \item \textbf{Open questions}: which generic replication cases can be
    performed one-way? That simplifies implementation a lot: ``Although most
    implementations use bidirectional communication, the communication from the
    update-receiver to the update-sender is just a (significant) performance
    optimization used to avoid redundant transfers of updates already known to
    the receiver. One-way convergence is also important in protocols that
    transmit updates via delay tolerant networks''\cite{b2}.
  \item \textbf{Open questions}: anything to propose for bidirectional state
    synchronization, without breaking the autonomy requirements? Alternatively,
    instead of bidirectionally replicating data, may an API-to-API based
    synchronization mechanisms become a replacement for such cases, like
    predictable cloud users/projects ID require no databases synchronized?
  \item \textbf{Open questions}: which of the existing causal consistent
    databases\cite{b6} or file systems\cite{b7} can be integrated as a global
    solution that fits all control/management cases for cloudlets and may
    benefit workloads consuming these as a service?
\end{itemize}

\section{Conclusion}
We defined autonomy requirements for multiple control/management planes of
massively distributed Fog environments and imposed operational invariants off
it. That brought us to consistency requirements for cloudlets state replication
and associated challenges. We introduced a replication topology building
concepts. Finally, we posed vision of key design principles, like queuing and
lazy replication, aliveness of the control and management planes and a unified
approach for the subject tooling. Possible solutions may be based on either
shared causal consistent databases, file systems, or client libraries that
replicate not shared local data low-level. Or as well client libraries that
replicate operations at an API high-level.

We want to position the unification principle as the most important thing and
the greatest opportunity for developers to do it ``the right way'', which is to
bring the best of two IaaS and PaaS worlds for end users whom such data
replication tooling might benefit as a service, i.e. cloud tenants,
infrastructure owners and operators. That also applies to any fog-based system
and is not limited to OpenStack or Kubernetes ecosystems.

Finally, we have to admit that because of outstanding future research work and
potentially huge amount of changes for IaaS/PaaS cloud components required to
implement or adopt either of the aforementioned options for multiple
control/management planes state replication, it perhaps should be a next
iteration after a minimum viable product (MVP) done as a single distributed
control plane separated from the management plane. That enables early
implementations for a very restricted subsets of operational invariants without
autonomy supported, but ``retain workloads operational as the best
effort''.

\section*{Acknowledgment}
Thanks to all who reviewed this paper, including but not limited to:
Reedip Banerjee, from Red Hat; Flavia Delicato and Paulo Pires, from Federal
University of Rio de Janeiro. Special thanks to all of the participants of
OpenStack Summit at Berlin, 2018.

\begin{thebibliography}{00}
\bibitem{b1} W. Lloyd, M. J. Freedman, M. Kaminsky, and D. G. Andersen, ``Don’t settle for eventual: Scalable causal consistency for wide-area storage with COPS,'' Proc. 23rd ACM Symposium on Operating Systems Principles (SOSP 11), Cascais, Portugal, October 2011.
\bibitem{b2} P. Mahajan, L. Alvisi, and M. Dahlin. ``Consistency, availability, and convergence,'' Technical Report TR-11-22, Univ. Texas at Austin, Dept. Comp. Sci., 2011.
\bibitem{b3} The Linux Foundation, ``Open Glossary of Edge Computing,'' [Online]. Available: \url{https://github.com/State-of-the-Edge/glossary}
\bibitem{b4} K. Kingsbury, ``Consistency Models,'' [Online]. Available: \url{https://jepsen.io/consistency}
\bibitem{b5} M. Bayer, ``Global Galera Database,'' [Online]. Available: \url{https://review.openstack.org/600555}
\bibitem{b6} M. M. Elbushra, J. Lindstrom, ``Causal Consistent Databases'', Open Journal of Databases (OJDB), Volume 2, Issue 1, 2015.
\bibitem{b7} PRCE (Projet de recherche collaborative — Entreprises), ``RainbowFS: Modular Consistency and Co-designed Massive File'' [Online]. Available: \url{http://rainbowfs.lip6.fr/data/RainbowFS-2016-04-12.pdf}
\end{thebibliography}
\end{document}
