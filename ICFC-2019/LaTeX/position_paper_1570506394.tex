% This work is licensed under the Creative Commons Attribution 4.0 International
% License. To view a copy of this license, visit
% http://creativecommons.org/licenses/by/4.0/ or send a letter to Creative
% Commons, PO Box 1866, Mountain View, CA 94042, USA.
\documentclass[conference]{IEEEtran}
\IEEEoverridecommandlockouts
% That is a public draft of a posistion paper
\usepackage{cite}
\usepackage{amsmath,amssymb,amsfonts}
\usepackage{algorithmic}
\usepackage{graphicx}
\usepackage{textcomp}
\usepackage{xcolor}
\usepackage{url}
\usepackage[linguistics]{forest}
\def\BibTeX{{\rm B\kern-.05em{\sc i\kern-.025em b}\kern-.08em
    T\kern-.1667em\lower.7ex\hbox{E}\kern-.125emX}}
\begin{document}

\title{DRAFT Position Paper: Edge Clouds Multiple Control Planes Data Replication
Challenges\\
}

\author{\IEEEauthorblockN{hidden for double-blind review purposes}
\IEEEauthorblockA{\textit{} \\
\textit{}\\
\\
}
}

\maketitle

\begin{abstract}
Fog computing is an emerging paradigm aiming at bringing cloud functions closer
to the end users and data sources. Traditional DC-centric DevOps paradigms
established for cloud computing should also span to Edge clouds, which expects
them being always manageable and available for monitoring/alerting, scaling,
upgrading and applying security fixes. That is a challenge as classic methods
do not fit edge cases. Data synchronization is also a common issue for
IaaS/PaaS platforms and Mobile Edge Computing environments hosted there. In
this paper, we aim at initiating discussions around that challenge. We define
operational invariants for Edge clouds based on the Always Available autonomy
requirement and related state-of-art work. We evaluate global/replicated causal
consistent data stores and middleware as possible match for meeting those
invariants. We point out a great opportunity for designing causal consistent
systems for edge cases as a common solution for cloud infrastructure owners,
DevOps/SRE and MEC applications. Finally, we bring vision of
Replication-as-a-Service, unified design approach to benefit applications, like
NFV/StatelessNF, and cloud middleware/APIs. Having RaaS implemented as
vendors-agnostic commodity software/storage systems that provide
interoperability over hybrid clouds and multiple service providers is the
ultimate mission for future work.
\end{abstract}

\begin{IEEEkeywords}
Open source software, Edge computing, Network function virtualization,
Distributed computing, System availability, Design
\end{IEEEkeywords}

\section{Introduction}
Hybrid and multi-cloud environments is inevitable future of cloud computing.
Interconnected private and public clouds, optionally hosting
Platform-as-a-Service (PaaS) solutions on top, like OpenShift/Kubernetes, with
its workloads pushed closer to end users, unlock great potential for Mobile
Edge Computing (MEC) and massively distributed scenarios. In fog environments,
expectations for management, control and operational capabilities are pretty
same as to the traditional cloud environments, while DC-centric approaches do
not work over wide area networks (WANs) and multiple autonomous control planes.
Whereby we can have state of applications or control/management planes
synchronized eventually and partially, which requires no strong consistency and
no global view for the most of the cases and most of the time. That imposes
requirements for specific distributed/replicated data stores and/or middleware
to maintain constant availability for centralized control/management planes
without violating the requirement of local manageability of
systems/applications. These must be available, while being offline or executing
a handover between base radio network stations, and recover its state fast upon
a crash. All that requires smart, WAN-optimized state transfers and resolving
of possible conflicts caused by concurrent updates or applying locally cached
operations after a network drop out ends. We aim at initiating debates on these
challenges through numerous communities, foundations and project groups
dedicated to building solutions for MEC, Network Function Virtualization (NFV)
and other edge computing cases that involve multiple control planes and
multi-site or multi-cloud operations.

\section{Background Concepts}
Aside of the established terms\cite{b3}, we define a few more for the data
processing and operational aspects:

\textbf{Deployment Data}: data that represents the configuration of
\textit{cloudlets}\cite{b3}, like API endpoints URI, or numbers of deployed
\textit{edge nodes}\cite{b3} in \textit{edge clouds}\cite{b3}. That data may
represent either the most recent state of a deployment, or arbitrary data
chunks/operations required to alter that state. When there is unresolved data
merging conflicts or queued operations pending for future execution, the most
recent state becomes the \textit{best known} one.

\textbf{Cloud Data}: similarly to deployment data, represents the most recent
or the best known internal and publicly visible state of cloudlets, like cloud
users or virtual routers. Cloud data also includes logs, performance and usage
statistics, state of message queues and the contents of databases. It may as
well represent either data chunks or operations targeted for some state
\begin{figure}[htbp]
\caption{Example Replication Topology}.
\begin{forest}
  [$\mathrm{EAL1}$
    [\textit{access edge layer}\cite{b3}
     [\textit{infrastructure edge}\cite{b3}
       [\textit{device edge}\cite{b3}]
     ]
    ]
    [$\mathrm{EAL2}$
      [$\mathrm{EAL3}$
        [$\mathrm{cloudlet1}$
        ]
      ]
      [$\mathrm{EAL4}$
        [$\mathrm{cloudlet2}$
          [$\mathrm{EAL5}$
            [$\mathrm{cloudlet3}$
          ]
          [$\mathrm{cloudlet4}$
          ]
        ]
      ]
    ]
  ]
]
\label{fig}
\end{forest}
\end{figure}

$\mathrm{S}$ transitioning into a new state $\mathrm{S'}$.

\textbf{Control Plane}: corresponds to any operations performed via cloudlets
API endpoints or command line tooling. E.g. starting a virtual machine
instance or a Kubernetes pod, or creating an OpenStack Keystone user. Such
operations are typically initiated by cloud applications, tenants or operators.

\textbf{Replication Topology}: represents a graph for allowed state
replication flows and targets for operations on interconnected cloudlets,
including \textit{edge aggregation layers}\cite{b3}. For hierarchical
topologies, there would be a limitation for horizontally interconnected
cloudlets cannot be replicating its state nor targeting operations to each
other. This corresponds to the acyclic graph (tree) topology. There may be
also P2P topologies\cite{b9} or mixed cases.

For the example tree (Fig.~\ref{fig}), the $\mathrm{infrastructure\ edge}$ can
replicate data, like contributing its local stats into the global view of
hardware and performance metrics, to the the edge aggregation layer
$\mathrm{EAL1}$ omitting the $\mathrm{access\ edge\ layer}$. Meanwhile the
latter can be pushing something, like deployment data changes, to the
$\mathrm{device\ edge}$ and $\mathrm{infrastructure\ edge}$. On the right side
of the graph, $\mathrm{cloudlet3}$ and $\mathrm{cloudlet4}$ cannot replicate to
each other, but can do to the edge aggregation layer $\mathrm{EAL1}$ either
directly or consequently via $\mathrm{EAL5}$, $\mathrm{cloudlet2}$,
$\mathrm{EAL4}$ and $\mathrm{EAL2}$, which is totally the replication
implementation specific. We also say that $\mathrm{EAL2}$ and $\mathrm{EAL1}$
are hierarchically upper associated with $\mathrm{EAL4}$ and $\mathrm{EAL5}$,
or just $\mathrm{EAL1}$ is an \textit{upper layer} for $\mathrm{EAL2}$, when we
mean that the latter is a potential subordinate of the former. Finally, we can
say that $\mathrm{EAL4}$ controls/manages $\mathrm{cloudlet2}$ and anything
sitting down of it.

That is, a control/manage-subordinate hierarchical association only defines
influence domains and does not imply state replication as a mandatory thing nor
imposes bidirectional data synchronization, but rather provides an opportunity
of one-way or bidirectional state replications and regulates a possibility of
issuing control/management operations. State replication may be performed from
subordinates to its upper associates, think of sending reports. Or vice versa,
think of sending directives to subordinates. While control/management
operations may only flow up-down, think of planning future work for
subordinates.

\textbf{Management Plane}: corresponds to administrative actions performed via
configuration and lifecycle management systems. Such operations are typically
targeted for cloudlets, like edge nodes, \textit{edge datacenters}\cite{b3},
or edge clouds. E.g. upgrading or reconfiguring cloudlets in a \textit{virtual
datacenter}\cite{b3}, or scaling up edge nodes. And typically initiated by
cloud infrastructure owners. For some cases, like Baremetal-as-a-Service,
tenants may as well initiate actions executed via the management plane.
Collecting logs, performance and usage statistics for monitoring and alerting
systems also represents the management plane operations, although it operates
with the cloud data.

When we refer to an edge aggregation layer and cloudlets under its
control/management, we mean exactly any operations executed via the
control/management planes of that edge aggregation layer and targeted for
cloudlets sitting down the nested connections graph. And a replication topology
regulates allowed targets for operations and state replication.

\textbf{Data Replication Conflict}: according to\cite{b1}, two operations on
the same target are in conflict if they are not related by causality.

\textbf{Always Available}: the operational mode of the control and management
planes that corresponds to \textit{sticky available causal
consistency}\cite{b4} data replication models, i.e. RTC (\textit{Real-Time
Causal}\cite{b2}), or \textit{causal+}\cite{b1}. Depending on the consistency
model choices, there may be additional constraints:
\begin{itemize}
  \item the stickiness property is a mandatory constraint for sticky available
    causal consistent systems. That is: ``on every non-faulty node, so long
    as clients only talk to the same servers, instead of switching to new
    ones''\cite{b4}.
  \item the real-time constraint is keeping the system time synchronized for
    all cloudlets. That is a mandatory constraint for RTC.
  \item \textit{one-way convergence}\cite{b2} is a mandatory for RTC.
\end{itemize}
Causal+ and RTC consistency ensure ordering of relative operations, i.e. all
causal related writes can be seen in the same order by all processes (connected
to the same server). It is also known that ``the causal consistency supports
non-blocking operations; i.e. processes may complete read or write operations
without waiting for global computation. Therefore, the causal consistency
overcomes the primary limit of stronger criteria: communication
latency''\cite{b6}. All that provides the best causal consistency guarantees we
can get for today for always available systems.

\section{Analysis and Discussion}
\subsection{Autonomy Requirements}
We define always available autonomy for cloudlets as the following strict
requirements:

\subsubsection{Maintain multiple control/management planes}
Autonomous cloudlets may only operate as always available when having multiple
control/management planes. For example, when an offline cloudlet cannot start
virtual machines or containers, that violates the autonomy requirements.

\subsubsection{Provide no read-only or blocking access for inter-cloudlet
operations}
Any operations performed on cloudlets state, despite its aliveness/failure
conditions as it's shown in the global view of upper edge aggregation layers,
fit consistency models that allow the involved control/management planes
operating as always available, and there is no limitations, like read-only or
blocking access. Operations may be queued for future processing in order to
meet this autonomy requirement though. Internal state of cloudlets is allowed
to keep its failure modes unchanged. E.g. its DC-centric quorum requirements
still apply for internal data transactions. That is a transitioning requirement
until the internal cloudlets state can be migrated to causal consistent data
storages as well.

\subsubsection{Provide a local control/management plane, whenever possible}
Operations on cloudlets can be scheduled at any given moment of time. For
offline/partitioned autonomous cloudlets, that can be done via local
control/management plane, if it exists and not failed. The same transitioning
considerations for internal cloudlets state apply.

\subsubsection{Support fully autonomous (offline) cloudlets}
Aggregation edge layer cloudlets should allow for arbitrary or all of its
managed/controlled cloudlets running fully autonomous long-time or indefinitely
long, queuing any operations targeted for such cloudlets. That is, to have
failure domains size of a 1. For a permanently disconnected cloudlet it may make
more sense though to detach it from its adjacent aggregation layer and/or
reorganize its place taken in the replication topology.

\subsubsection{Queue operations to keep it always available at the best
effort}
Queued operations have to be eventually replayed if/after the
control/management plane capabilities restored, or dropped (e.g. expired)
otherwise. That poses a delayed replication principle. If there is intermediate
aggregation edge layers down the way to the target of the queued operations,
the queue may be shared across each of the involved aggregation edge layers or
optionally, queued operations may be distributed across not shared queues. That
should reduce the associated memory and disc pressure for aggregation layers.
Note that we induced no ordering constraints for the operations replayed from
queues, that should be defined on case-by-case basis and ideally maintain
ordering of causal related operations only, like create a VM and snapshot that
VM. Global unique IDs may be a good fit for maintaining such a causal ordering.

\subsubsection{Provide a global view for cloudlets aliveness state}
Global view of cloudlets needs to be periodically presented for at least one of
upper aggregation edge layers. For example, with
the state marks, like ``unknown'', or ``autonomous''; ``synchronizing'',
``connected''; ``failed'', or ``disconnected'', or ``fenced'', if and only if
it is confirmed as failed, or manually disconnected, or fenced automatically.
That poses the aliveness of the control/management planes principle.

\subsection{Operational Invariants}
To be always available as we defined it, control and management planes of
cloudlets should provide the following operational capabilities
(\textit{invariants} hereafter):

\subsubsection{Keep control planes always available at best effort}
CRUD (Create, Read, Update and Delete) operations on cloud data can always be
requested via API/CLI of local cloudlets or upper layers. The same queuing
requirements apply as it is defined for the autonomy requirements.

\subsubsection{Do not wait for edge aggregation layers control planes for local
operations}
Local CRUD operations for offline cloudlets, if its control plane exists and
not failed, can be processed without waiting for the upper aggregation layers
to recover its control over the cloudlets. When a cloudlet is running only
compute/storage resources, it cannot meet this requirement.

\subsubsection{Allow local scaling of infrastructure edge nodes without waiting
for management planes of upper aggregation layers}
Similarly the to control plane operations, deployed infrastructure edge nodes
can always be scheduled for scaling up/down by the cloudlets local management
planes, if it is possible (a cloudlet may be relying on the remote
configuration management only), or via its upper layers. Same queuing
requirements apply for operations.

\subsubsection{Allow hotfixes and kernel/software updates applied locally for
cloudlets}
Security patches and minor system updates, including kernel upgrades, can
always be scheduled for installation by the same meanings (via operations
issued locally or by the associated aggregation layers, including the same
queuing requirements).

\subsubsection{Allow major software versions upgrades applied locally for
cloudlets}
Similarly, major versions of system components, like OpenStack or
OpenShift/Kubernetes platforms, can be always scheduled for upgrades, using the
same meanings as above.

\subsubsection{Provide an extended global view for cloudlets}
Additionally to the aforementioned global view for cloudlets control/management
plane aliveness state marks, there needs to be a periodically updated global
view for each of the edge aggregation layers into its controlled/managed
cloudlets, at least the adjacent ones, for the key system administration
aspects, like hardware status, power management, systems state logging,
monitoring and alerting events, performance and metering statistics.

\subsection{State Replication Consistency Requirements}
As it follows from the defined always available autonomy requirements and
operational invariants, we define the following data replication
requirements\footnote{when we refer to \textit{data} or \textit{state}, we
do not differentiate either that is deployment/cloud data, internal state of
an NFV application or cloud API operations. That poses the \textbf{unified approach principle}}:

\subsubsection{Incorporate convergent conflict handling\cite{b1}}
Data replication conflicts can be resolved automatically, or by hand and
maintained as causal related. The conflicts resolving strategies and rules
should be customizable, like ``last writer wins'' or ``return them all''. After
the conflicts resolved, the data may be considered causal related, that is by
definition\cite{b1} of the data conflicts in eventually consistent systems.

\subsubsection{Prefer one-way convergence in replication topologies}
As far as the replication topology and queuing capabilities allow
that, causal related data can be replicated across cloudlets. Prefer one-way
convergence and avoid bidirectional replication whenever possible.

\subsubsection{Bidirectional replication is only a nice to have requirement}
Bidirectional (two-way convergence) data replication is not a strict
requirement but is nice to have. Indeed, some state needs to be replicated
one-way from aggregation edge layer to cloudlets under its control/management,
like virtual machine or hardware provisioning images data. While logs,
performance and metering statistics may be collected only from cloudlets to its
upper layers.

OpenStack/Kubernetes, have yet support for neither causal consistent storage
backends for its cloud/deployment data, nor middleware that could drive
replication of casual related
state\footnote{StarlingX:\url{https://storyboard.openstack.org/#!/story/2002842}
builds on top of the current strongly consistent database backend}. OpenStack
cloud data is stored in a distributed database\footnote{See an evaluation of
CockroachDB:\url{https://www.cockroachlabs.com/docs}
strongly consistent KVS data store:\url{https://github.com/BeyondTheClouds/juice}} based on
\textit{unavailable}\cite{b4} strongly consistent models, e.g.
\textit{serializable}\cite{b4}. In OpenShift/Kubernetes clusters state is
backed with a KVS data store, which also supports only the strong consistency.
These does not scale across datacentres, which poses an open opportunity for
designing causal consistent state replication systems.

From the other side, the \textit{total available}\cite{b4} consistency models
weaker than causal or eventual, leave the programmers of end systems to deal
with \textit{fuzzy reads}\cite{b4}, \textit{phantoms}\cite{b4}, discarded
write-only transactions or empty state returned for any reads.

\subsection{Data Replication Challenges}
All that brings us to challenges that need to be addressed for multiple
control/management planes:
\begin{itemize}
  \item identifying types of control/management operations and replicated
    data, when grouping those into particular replication topologies. Such
    groups may be identified by multiple metrics, like communication latency,
    tolerated duration of network partitions for offline cloudlets, one- or
    two-way convergence based replication of either low-level data or
    operations at the higher API-to-API levels.
  \item the unified replication topology should not bring excessive operational
    overhead, like maintaining all of the identified types of operations
    simultaneously for the end system, and require not too much of human care.
  \item picking the right replication topologies for each of the involved
    system components, like an identity provider or images streaming services.
    For example, would a one-way convergent per-key causality (provided by
    distributed systems, like
    Riak\footnote{\url{https://docs.riak.com/riak/kv/latest/learn/concepts/}}
    or Cassandra\footnote{\url{https://github.com/wlloyd/eiger}}),
    fit the data replication needs for a distributed cloud identity service? Or
    should caching of images maintain a peer-to-peer mesh topoligy replicated
    across cloudlets?
  \item programming the conflicts-free shims or ``smart'' conflicts resolvers
    based on the picked replication topologies. The ``last writer wins'' may be
    a good fit for some basic cases for objects stored in distributed
    databases/KVS, while ``return them all'' could benefit the more advanced
    cases, but might require assistance of artificial intelligence.
  \item keeping the state replication topologies always efficient and adjusting
    itself dynamically. E.g., when executing a handover for a mobile subscriber
    in a 5G network, an orchetrator must identify the adjacent endpoints based
    on the subscriber location and define the numbers of replicas to place
    there. Or distributing the queued control plane operations targeted for the
    associated offline cloudlets might require more of the nested edge
    aggregation layers or peers to be added dynamically or statically, when the
    load exceeds hardware capabilities of a particular aggregation site.
  \item programming middleware that abides the unified approach and fits the
    targeted Replication-as-a-Service (RaaS) cases for IaaS/Paas control and
    management planes as well as the edge-native, like NFV, workloads. For the
    most of the cases, one size does not fit all, so future RaaS solutions
    should be tightly coupled with its suggested replication topologies.
\end{itemize}

\begin{itemize}
  \item \textbf{Open questions}: what of the known Edge computing cases common
    for OpenStack IaaS (and/or Kubernetes PaaS) and NFV applications, can be
    expressed via one-way converged data replication, which simplifies
    implementation of end-to-end solutions? For the remaining two-way
    convergent cases, may an API-to-API based synchronization fully
    address the needs for bidirectional data replication?
\end{itemize}

\subsection{Vision of a Unified Deployment/Cloud State Replication Design}
The definition we made for always available distributed systems self-explains
why the causal consistent state replication is the best match for the
massively distributed cloudlets autonomy requirements and operational
invariants as we defined those.

The vision of the unified architecture for future state replication tooling
imposes it should be solving the multiple control/management planes data
synchronization problems for IaaS, PaaS and end users consuming it as RaaS.
Although generic version control systems, like Git, might fit all cases for
deployment data replicating and conflicts resolving, that would break the
unified design approach for cloud data/state replication.

Client libraries implementing causal consistent data replication and
customizable conflicts resolving rules may provide a unification layer for
different underlying databases/KVS (Key Value Storage). The replication will be
effectively acting as database/KVS-to-database/KVS data synchronization tooling
syncing data at a database level. The main benefit for such an approach is no a
shared data storage needed. Instead, the underlying local to cloudlets data
storages may keep operating as is, share nothing and provide unavailable
consistency models stronger than causal consistency. And cloudlets may keep
using different solutions for its local data storages as far as the state
replication tooling may support such backends.

Additionally, client libraries may replicate not only data but operations at an
API level as well, i.e. resolve conflicting operations on-fly, then apply the
resulting causal related operations for its original targets, effectively
replicating changes at an API level. Operations queued by the
control/management planes, including those targeted for offline cloudlets, may
be also processed that way.

\section{Related Work}
COPS\cite{b1} provides theoretical fundamentals for causal+ consistency and
focuses on highly scalable middleware libraries implementing causal consistent
data operations. The similar approach is taken for Indigo middleware\cite{b10}
that gives strong fundamentals on creating application-centric programming
methodologies that leverage invariant-repair/violation-avoidance techniques and
rely on immutable data structures (CRDTs). Ultimately, that should help
programmers to enforce Explicit Consistency by extending existing applications
logic and building middleware libraries that operate on top of the underlying
causal consistent storage backends. That is, like Walter\cite{b11} or
SwiftCloud\cite{b12}.

Walter\cite{b11} KVS uses a set-like CRDTs. It provides a strong consistency
guarantee within a site and causal ordering across sites. It introduces
Parallel Snapshot Isolation (PSI) property that extends snapshot isolation by
allowing different sites to diverge with different commit orderings. This is
also known as sticky available causal consistency model, where applications
should avoid changing its connection endpoints and maintain sticky sessions.
Walter performs asynchronous replication across multiple sites and supports
partial replication for multi-cluster scenarios to address scalability
bottlenecks.

SwiftCloud distributed object database\cite{b12} brings geo-replication to the
client machines instead. It supports interactive transactions that span
multiple CRDT types. To its authors' knowledge, asynchronous replication
systems ensure fault-tolerant causal consistency only within the boundaries of
the DC, while SwiftCloud guarantees convergent causal consistency all the way
to resource-poor end clients. SwiftCloud also proposes a novel client-assisted
failover mechanism that trades latency by a small increase in staleness of
data.

Global/stretched Causal Consistent Databases\cite{b6} work presents potential
applications and databases that could use the causal consistency and shows
possible methods to implement that model. It also compares serializability,
eventual and causal consistency using a running example. There is an important
conclusion that to the authors’ knowledge there is no commercial or mature
systems using the causal consistency.

Bolt-on\cite{b13} shim takes another approach and allows to leverage existing
production-ready eventually consistent data stores virtually upgrading it to
provide convergent causal consistency.

STACK Research Group\cite{b8} provides a list of the features required to
operate and use edge computing resources. The listed requirements are
complementary to this work and represent the operational invariants approached
from OpenStack developers and operators (DevOps) angle, the view point that
also covers interoperability between multiple operators. The latter is an
important requirement for hybrid clouds and NFV Edge cases, like Virtual
Customer Premises Equipment (vCPE).

In Mobile Edge Computing environments, there is also high demand for novel
lightweight data replication and applications live migration solutions. Those
must perform well over WAN and not being DC-centric. Proactive data replication
techniques to cope with user mobility considered in the related work\cite{b14}.
It poses challenges of efficient scheduling of data replicas over the edge
nodes. According to ETSI reference architecture, MEC Orchestrators are
responsible for solving these problems via user mobility prediction, while
virtualization infrastructure management (VIM) should implement the proposed
procedures of proactive migration. It is notable that container‐based VIMs are
considered the most promising solutions for MEC environments, mostly due to
reduced footprint of containers. Virtualized 5G in Evolved Packet Core (vEPC)
Architectures\cite{b15} serves another good example of high demand emerging
from the world of Telcos. The work describes a state sharing mechanism across
different datacenters that leverages Edge Synchronization Protocol(ESP) and
Abstract Syntax Notation.1 (ASN.1). None of these two works mention
causal consistency but it reads between the lines as the such, that answers the
questions, like which state portions should be replicated, to which cloudlets
and under which conditions. E.g. predicted users' trajectories or operations
involving particular mobile subscribers may help to establish causal relations
and produce ultimate replication decisions at the applications level.
StatelessNF\cite{b16} rearchitects NFV applications to decouple its internal
dynamic state, like connections tracked by firewalls, into a low-latency DRAM
storage tier, like one provided by the underneath VIM. StatelessNF relies on
RAMCloud\footnote{\url{https://ramcloud.stanford.edu}}, where all data is
stored in DRAM. That provides 100-1000x lower latency than disk-based systems
and 100-1000x greater throughput, which greatly reduces possibility of
concurrent updates causing non-mergable data conflicts. While this benefits NFV
cases a lot, RAMCloud may not meet well the defined autonomy requirements for
control/management planes, where concurrent data updates and cross-datacenter
replication is inevitable. In such setups, RAMCloud as a potential RaaS
solution has no more its low-latency advantages.

A. Lebre et al.\cite{b9} bring vision of a Peer-to-Peer (P2P) OpenStack IaaS
Manager, which is opposed to hierarchical distributed topologies. It is stated
that P2P saves maintenance costs associated with hierarchies, while the latter
also require complex operations in case of failures. The unified IaaS Manager
is positioned as an alternative to stretched control plane and federated
clouds. It emphasizes on challenges of data locality, efficient cloud storage,
interoperability peering agreements, autonomous lifecycle tooling and new
edge-native cloud APIs as paramount requirements. Such a view-point naturally
complements the vision of unified management plane as we introduced it for this
paper.

The edge-cloud native APIs may be also designed with the approaches that
Indigo\cite{b10} or RainbowFS\cite{b7} takes. The latter work focuses on
Just-Right (modular) Consistency and flexible service level agreements, like
latency requirements, enforced data locality and smart storage placement
strategies unique to Fog/Edge platforms and dictated by applications designed
for it on case-by-case basis. ``Whereby an application pays only the
consistency price that it strictly requires''\cite{b7}.

\begin{itemize}
  \item \textbf{Open questions}: are there open-source projects to benefit
    OpenStack/Kubernetes (and NFV apps) for edge cases that fit into multiple
    autonomous sites? Either mature ones or in active development,
    implementing a causal consistent database/KVS, like Walter/SwiftCloud, or
    midleware/shims operating on top of such weakly consistent systems, like
    RTC/COPS/Indigo/Bolt-on? For example, SwiftCloud is an open-source project
    available under Apache 2.0 license and might be a good start for future
    work of adopting it for OpenStack/Kubernetes and edge-native applications
    as RaaS. Combining it with RAMCloud might end up in a powerful NFV option.
  \item \textbf{Open questions}: what of the stretched/global clusters can
    support multi-site failure events, assuming at the edge there may be a lot
    of small failure domains? And which of those can provide two-way
    convergent causal consistent data replication across edge sites?
\end{itemize}

\section{Conclusion}
We defined autonomy requirements for multiple control/management planes of
massively distributed Fog environments and imposed operational invariants off
it. That brought us to consistency requirements for cloudlets state replication
and associated challenges. We introduced a replication topology building
concepts. We posed vision of key design principles, like queuing and delayed
replication, aliveness of the control and management planes and a unified
approach for the middleware tooling. Possible subject solutions may be based on
either of global/stretched causal consistent data stores, or DC-centric shims
that operate on top of eventually or causal consistent replicated autonomous
data stores. Or client-side middleware placed at the edge. That is all to
synchronize data low-level. Alternatively, that may be middleware that
replicates operations at an API-to-API high-level.

We want to emphasize on the unification principle as the great opportunity for
developers to bring the best of IaaS and PaaS worlds for end users whom such
data replication tooling might benefit as a service, i.e. DevOps and tenants
workloads, like MEC environments and NFV. It is also an opportunity for Telco
side to leverage the commodity replication services of infrastructure layers
for the hosted virtual network functions (VNFs). And StalessNF benefits from
such common RaaS solutions the most. The unification principle also applies to
any fog-based system and is not limited to OpenStack, Kubernetes or NFV
ecosystems. We only focus on those as a good starting point for the future
work.

To draw the line for where/when future work starts, we should remember
that ``retain workloads operational as the best effort'' principle provides a
good start for minimum viable products (MVP), like Distributed Compute Node
(DCN) scenario. It is also known to be highly wanted by Telco segment for its
next-generation Virtual Radio Access cellular Networks (vRAN).

A centralized control/management plane enables early implementations for DCN as
it has no autonomy requirements. While post-MVP phases should be evaluated for
the most advanced cases, like MEC vEPC, IoT, AI, autonomous aerial drones, big
data processing at the edge networks etc. That is where the data replication
challenges to fully arise for expensive handover procedures, mobile networks'
subscribers state transfers and just generic Day-2 operations of multiple
control/management planes of Edge clouds.

\section*{Acknowledgment}
Thanks to all who reviewed this paper, including but not limited to:
Reedip Banerjee, from Red Hat; Flavia Delicato and Paulo Pires, from Federal
University of Rio de Janeiro. Special thanks to all of the participants of
OpenStack Summit at Berlin, 2018.

\begin{thebibliography}{00}
\bibitem{b1} W. Lloyd, M. J. Freedman, M. Kaminsky, and D. G. Andersen, ``Don’t settle for eventual: Scalable causal consistency for wide-area storage with COPS,'' Proc. 23rd ACM Symposium on Operating Systems Principles (SOSP 11), Cascais, Portugal, October 2011.
\bibitem{b2} P. Mahajan, L. Alvisi, and M. Dahlin. ``Consistency, availability, and convergence,'' Technical Report TR-11-22, Univ. Texas at Austin, Dept. Comp. Sci., 2011.
\bibitem{b3} The Linux Foundation, ``Open Glossary of Edge Computing,''
  [Online]. \newline Available: \url{https://github.com/State-of-the-Edge/glossary}
\bibitem{b4} K. Kingsbury, ``Consistency Models,'' [Online]. Available: \newline \url{https://jepsen.io/consistency}.
\bibitem{b5} M. Bayer, ``Global Galera Database,'' [Online]. Available: \newline \url{https://review.openstack.org/600555}.
\bibitem{b6} M. M. Elbushra, J. Lindstrom, ``Causal Consistent Databases'', Open Journal of Databases (OJDB), Volume 2, Issue 1, 2015.
\bibitem{b7} PRCE (Projet de recherche collaborative — Entreprises), ``RainbowFS: Modular Consistency and Co-designed Massive File'' [Online]. Available: \url{http://rainbowfs.lip6.fr/data/RainbowFS-2016-04-12.pdf}.
\bibitem{b8} R. A. Cherrueau, A. Lebre, D. Pertin, F. Wuhib, J. M. Soares, ``Edge Computing Resource Management System: a Critical Building Block! Initiating the debate via OpenStack'', USENIX HotEdge’18 Workshop, 2018.
\bibitem{b9} A. Lebre, J. Pastor, A. Simonet, F. Desprez, ``Revising OpenStack to Operate Fog/Edge Computing Infrastructures'', IEEE International Conference on Cloud Engineering (IC2E), 2017.
\bibitem{b10} V. Balegas, N. Preguiça, R. Rodrigues, S. Duarte, C. Ferreira, M. Najafzadeh, M. Shapiro, ``Putting Consistency back into Eventual Consistency'', Euro. Conf. on Computing Systems (EuroSys), Bordeaux, France, 2015.
\bibitem{b11} Y. Sovran, R. Power, M. Aguilera, and J. Li, ``Transactional Storage for Geo-replicated Systems'', SOSP'11, page 385-400. New York, USA, 2011. ACM.
\bibitem{b12} M. Zawirski, A. Bieniusa, V. Balegas, S. Duarte, C. Baquero, M. Shapiro, and N. Preguica, ``SwiftCloud: Fault-Tolerant Geo-Replication Integrated all the Way to the Client Machine'', Research Report RR-8347, INRIA, Oct. 2013.
\bibitem{b13} P. Bailis, A. Ghodsi, J. M. Hellerstein, and I. Stoica, ``Bolt-on Causal Consistency'', SIGMOD'13, pages 761–772, New York, USA, 2013. ACM.
\bibitem{b14} I. Farris, T. Taleb, H. Flinck, A. Iera, ``Providing ultra‐short latency to user‐centric 5G applications at the mobile network edge'', Transactions on Emerging Telecommunications Technologies, Vol. 29, No. 4, e3169, 2018.
\bibitem{b15} E. Cau, M. Corici, P. Bellavista, L. Foschini, G. Carella, A. Edmonds, and T. Bohnert, ``Efficient Exploitation of Mobile Edge Computing for Virtualized 5G in EPC Architectures'', MobileCloud, page 100-109. IEEE Computer Society, 2016.
\bibitem{b16} M. Kablan, A. Alsudais, E. Keller, Franck Le,``Stateless Network Functions: Breaking the Tight Coupling of State and Processing'', NSDI, 97-112, 2017. 44, 2017.
\end{thebibliography}
\end{document}
