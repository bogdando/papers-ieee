\documentclass[conference]{IEEEtran}
\IEEEoverridecommandlockouts
% The preceding line is only needed to identify funding in the first footnote. If that is unneeded, please comment it out.
\usepackage{cite}
\usepackage{amsmath,amssymb,amsfonts}
\usepackage{algorithmic}
\usepackage{graphicx}
\usepackage{textcomp}
\usepackage{xcolor}
\usepackage{url}
\def\BibTeX{{\rm B\kern-.05em{\sc i\kern-.025em b}\kern-.08em
    T\kern-.1667em\lower.7ex\hbox{E}\kern-.125emX}}
\begin{document}

\title{Edge Clouds Control Plane and Management Data Consistency
Challenges: Position Paper for IEEE International Conference on Cloud
Engineering, 2019\\ }

\author{\IEEEauthorblockN{Bohdan Dobrelia}
\IEEEauthorblockA{\textit{OpenStack platform} \\
\textit{Red Hat}\\
Poznan, Poland \\
bdobreli@redhat.com}
}

\maketitle

\begin{abstract}
Fog computing is emerging Cloud of (Edge) Clouds technology. Its control plane
and deployments data synchronization is a major challenge. Autonomy requirements
expect even the most distant edge sites always manageable, available for
monitoring and alerting, scaling up/down, upgrading and applying security fixes.
Whenever temporary disconnected sites are managed locally or centrally, some
changes and data need to be eventually synchronized back to the central site(s)
with having its merge-conflicts resolved for the central data hub(s). While
some data needs to be pushed from the central site(s) to the Edge, which might
require resolving data collisions at the remote sites as well. In this paper,
we position the outstanding data synchronization problems for OpenStack
platform becoming a cloud solution number one for fog computing. We define the
inter-cloud operational invariants based on that Always Available autonomy
requirement. We show that a causally consistent data storage is the best
match for the outlined operational invariants and there is a great opportunity
for designing such a solution for Edge clouds. Finally, the paper brings the
vision of unified tooling to solve the data synchronization problems the same
way for infrastructure owners, IaaS cloud operators and tenants running
workloads for PaaS, like OpenShift or Kubernetes deployed on top of Edge
clouds.
\end{abstract}

\begin{IEEEkeywords}
Open source software, Edge computing, Distributed computing, System
availability, Design
\end{IEEEkeywords}

\section{Introduction}

OpenStack is an Infrastructure-as-a-Service platform number one for private
cloud computing, and it becomes being so for fog computing as well.
Hybridization and Mutli-cloud trends for private clouds interconnected with
public clouds and Platform-as-a-Service (PaaS) solutions, like
OpenShift/Kubernetes, allow the containerization of micro-services oriented
workloads to emerge in a highly portable, self-contained and the hosting
cloud-agnostic way. Giving it massively distributed scale of fog computing and
bringing the data it operates closer to end users, opens great opportunities
for Internet of Things (IoT) and nextgen global telecommunication technologies,
which first of all requires low-latency and higlhy responsive interfaces always
available for end users.

Speaking of always available, back to the system administration realities, the
Edge clouds control and management plane capabilities in such a perfect world
shall not fall behind as well. This paper is about to position the associated
data replication challenges and to bring vision of future development trends on
that topic, both for OpenStack and hopefully for anything residing on top of
it, e.g. PaaS and/or workloads designed for massively distributed scale and
following the IoT/fog computing best practices.

\section{Glossary}

Aside of the established terms\cite{b3}, we define a few more for the data
processing and operational aspects:

\textbf{Deployment Data}: data that represents the configuration of
\textit{cloudlets}\cite{b3}, like API endpoints URI, or numbers of deployed
\textit{edge nodes}\cite{b3} in \textit{edge clouds}\cite{b3}. That data
represents the most recent state of a deployment.

\textbf{Cloud Data}: represents the most recent\footnote{when there is
unresolved data merging conflicts, the most recent state becomes the best known
state} internal and publicly visible state of cloudlets, like cloud users or
virtual routers. Cloud data also includes logs, performance and usage
statistics, state of message queues and the contents of databases.

\textbf{Control Plane}: corresponds to any operations performed via cloudlets
API endpoints or command line tooling. For example, starting a virtual machine
instance, or creating a cloud user. Such operations are typically initiated by
cloud applications, tenants or operators. When we refer to an \textit{edge
aggregation layer}\cite{b3} and cloudlets under its control, we mean exactly
any operations executed via the control plane of that edge aggregation layer
and targeted for cloudlets sitting as the next hop graph connection. If the
next hop is represented by another edge aggregation layer, the same applies to
its adjacent cloudlets. So effectively there is no next hop only limitations
for control plane actions propagated over the nested levels of cloudlets.

\textbf{Management Plane}: corresponds to administrative actions performed via
configuration and lifecycle management systems. Such operations are typically
targeted for cloudlets, like edge nodes, \textit{edge data centers}\cite{b3},
or edge clouds. E.g.  upgrading or reconfiguring cloudlets in a \textit{virtual
data center} \cite{b3}, or scaling up edge nodes. And typically initiated by
cloud infrastructure owners. For some cases, like Baremetal-as-a-Service,
tenants may as well initiate actions executed via the management plane.
Collecting logs, performance and usage statistics for monitoring and alerting
systems also represents the management plane operations, although it operates
with the cloud data. When we refer to an \textit{edge aggregation
layer}\cite{b3} and cloudlets under its management, we mean exactly
administrative/deployment related operations executed via the management plane.
Similarly to the control plane, we impose no nesting limitations.

\textbf{Always Available}: the operational mode of the control and management
planes that corresponds to \textit{sticky available causal
consistency}\cite{b4} data replication models, i.e. RTC (\textit{Real-Time
Causal}\cite{b2}), or \textit{causal+}\cite{b1}. Depending on the consistency
model choices, there may be additional constraints:

\begin{itemize}
  \item the stickiness property is a mandatory constraint for sticky available
    causally consistent systems. That is: ``on every non-faulty node, so long
    as clients only talk to the same servers, instead of switching to new
    ones''\cite{b4}.
  \item the real-time constraint is keeping the system time synchronized for
    all cloudlets. That is a mandatory constraint for RTC.
  \item \textit{one-way convergence}\cite{b2} is a mandatory for RTC.
\end{itemize}

Causal+ and RTC consistency ensures ordering of relative operations, i.e.
all causally related writes can be seen in the same order by all processes
(connected to the same server). All that provides the best causal consistency
guarantees we can get for today for always available systemts.

\section{Analysis and Discussion}

\subsection{Autonomy Requirements}

We define always available autonomy for cloudlets as the following strict
requirements:

\begin{itemize}
  \item any operations performed on cloudlets state\footnote{despite the
    cloudlets aliveness or failure conditions} fit data consistency models that
    allow the involved control/management planes operating as always available,
    and there is no limitations, like read-only or blocking access.
  \item cloudlets data can be modified at any given moment of time, despite of
    inter-cloudlets network connectivity\footnote{for disconnected/partitioned
    cloudlets, data can be modified via local control/management plane, if
    exists and not failed. Despite the adjacent aggregation edge
    layer global view and/or quorum requirements}.
  \item aggregation edge layer cloudlets allows all of its managed/controlled
    cloudlets running fully autonomous long-time, having all the outgoing
    operations queued and either eventually applied with conflicts resolved, or
    dropped/expired.
  \item the global view of fully autonomous cloudlets needs to be represented
    for the agregation edge layer. For example, with the state marks, like
    ``unknown/autonomous'', ``synchronizing'', ``connected'',
    ``failed/disconnected/fenced'', if and only if it is confirmed as failed,
    or manually disconnected, or fenced automatically.
\end{itemize}

\subsection{Operational Invariants}

To be always available as we defined it, control and management planes of
cloudlets should provide the following operational capabilities
(\textit{invariants} hereafter):

\begin{itemize}
  \item TBD (see ./ICFC-2019/challenges.md)
\end{itemize}

\subsection{Data Replication Consistency Requirements}

The operational invariants dictate inevitable presense of shared state and
sophisticated data replication mechanisms\footnote{when we refer to just
\textit{data} or \textit{state}, we do not differentiate either that is
deployment or cloud data, which poses the unified approach principle} among
cloudlets.

As it follows from the defined always available autonomy requirements and
operational invariants, we define the following data replication requirements:

\begin{itemize}
  \item data can be replicated eventually across cloudlets and the adjacent
    aggregation edge layer, but not horizontally\footnote{also implies there is
    no horizontal data replication across neighbor aggregation edge layers.
    This corresponds to the acyclic graph (tree) topology. Also, only one-way
    data replication may be possible in case of an RTC consistent data
    storage}.
  \item bidirectional (two-way convergence) data replication is not a strict
    requirement but is nice to have. Indeed, some state needs to be replicated
    one-way from aggregation edge layer to cloudlets under its
    control/management, like virtual machine or hardware provisioning images
    data. While logs, performance and metering statistics may be collected only
    from cloudlets to the adjacent aggregation edge layer.
  \item data replication conflicts can be resolved automatically or by hand,
    and/or queued\footnote{depending on the numbers and allowed isolation
    periods of cloudlets under control/management, the disc and memory
    requirements for aggregation edge layers may vary drastically} for later
    processing.
\end{itemize}

OpenStack and OpenShift/Kubernetes, have yet causally consistent data
backends\footnote{that is, for cloud/deployment data only} supported.
OpenStack cloud data is normally stored in databases via transactions based on
stronger than causal \textit{unavailable}\cite{b4} data consistency models,
e.g.  \textit{serializable}\cite{b4}, or \textit{repeatable read}\cite{b4}.

The weaker than causal+ and RTC \textit{total available}\cite{b4}
consistency models may be considered as an alternative. Transactional global
databases\cite{b5}, may technically support it\footnote{not Galera/MariaDB
cluster though, as it has a strict quorum requierements for database writes}. A
weaker consistency model provides a really poor alternative though as it brings
increased implementaion complexity, like corner cases handling for either the
storage replicas, or client sided, or both, associated with relaxed
constraints. E.g. \textit{monotonic atomic view}\cite{b4} does not impose any
real-time constraints, while RTC does, which somewhat simplifies
the end system design. Additionally, monotonic atomic view would require
sophisticated handling of \textit{fuzzy reads}\cite{b4},
\textit{phantoms}\cite{b4}, discarded write-only transactions, empty state
returned for any reads. All that makes that the strongest option for totally
available data backends less preferable than causally consistent ones.

Kubernetes clusters state is backed with Etcd, which only supports the stronger
than RTC consistency models.

\subsection{Vision of a Unified Deployment/Cloud Data Storage Design}

The definition we made for always available distributed systems self-explains
why the causally consistent storage backends is the best match for the
cloudlets autonomy requirements and operational invariants as we defined those.

The vision of the unified architecture based on such an always available data
storage dictates us to not consider different backends for cloud and deployment
data. Although generic version control systems, like Git, might fit all cases
for deployment data versioning, replicating and conflicts resolving, that would
break the unified design approach for cloud data replication.

Client libraries implementing causally consistent data replication and
customizable conflicts resolving rules may provide a unification layer for
different underlying databases/KVS (Key Value Storage). The replication will be
effectively acting as database/KVS-to-database/KVS data synchronization
tooling. The main benefit for such an approach is no a global data storage
needed. Instead, the underlying local to cloudlets data storages may keep
operating as is, share nothing and provide unavailable consistency models
stronger than causal consistency. And cloudlets may keep using different
solutions for its local data storages as far as the tooling supports such
backends.

Additionally, client libraries may replicate not only data but operations as
well, i.e. resolve conflicting operations on-fly or picking from a queue, then
apply the resulting causally related operations for its original targets,
effectively acting as API-to-API retranslators. Any operations queued by
control/management planes, including those targeted for disconnected/autonomous
cloudlets, may be also processed that way.

COPS formally proves implementation of a client library and highly scalable
tooling for causal+ data operations. By design, it does not impose any
real-time constraints and supports a single edge data center failure. The real
tooling made off that base, may be operating on top of the nonshared local
cloudlets databases, or KVS, that provide the stronger consistency guarantees
by the costs of reduced local availability\footnote{that is, the local view for
a cloudlet and have no impact onto global views}. That would work as weaker
consistency guarantees work well, when built on top of the stronger ones, and
provide an always available global view of cloudlets for the adjacent
aggregation edge layer. Replicating the state changes via causally related
operations and conflicts resolving via custom handlers is that COPS covers as
well.

Global causally consistent databases\cite{b6} is an alternative approach to
client libraries operating on top of cloudlets databases/KVS/API.  The downside
is such a database has to be supported as a control/management planes data
backend for OpenStack/OpenShift/Kubernetes and/or any stateful cloud
applications leveraging such a global database as a Replication-as-a-Service
solution. And local cloudlets databases/KVS have to be switched to that global
database.

\textbf{Open questions}:
\begin{itemize}
  \item Does COPS retains operations causal+ related when executed over
    multiple datacenters failure events (or extended time of being network
    partitioned?)
  \item Is COPS applicable for two-way convergent systems, in terms of
    \cite{b2}, for bidirectional causal+ replications? Given that \cite{b1}
    proves RTC provides the strongest causal consistency for one-way
    convergence only, and given that it provides stronger consistency than
    causal+, we can conclude that causal+ cannot provide the strongest causal
    consistency for bidirectional communication neither.
  \item How much of all of the cloudlets data replication cases can be
    performed one-way? That would drastically simplify future implementation:
    ``Although most implementations use bidirectional communication, the
    communication from the update-receiver to the updatesender is just a
    (significant) performance optimization used to avoid redundant transfers of
    updates already known to the receiver. One-way convergence is also
    important in protocols that transmit updates via delay tolerant
    networks''\cite{b2}.
  \item What solutions can we propose for operations that still do require a
    bidirectional data synchronization, like unique cloud users ID, without
    breaking the always availability autonomy requirements for the control
    and management planes? Global causally consistent databases may be
    a good choice for that. Alternatively, instead of bidirectionally
    replicating such data, inter-cloudlets API-to-API based synchronization
    mechanisms may become a solution.
\end{itemize}

TODO: find a use for RTC and \cite{b6} alternatives to form more
options for vision. Finally, make preferences for causal databases vs KVS, if
possible?

\section{Related work}

TBD maybe?

\section{Conclusion}

TBD

\begin{thebibliography}{00}
\bibitem{b1} W. Lloyd, M. J. Freedman, M. Kaminsky, and D. G. Andersen, ``Don’t settle for eventual: Scalable causal consistency for wide-area storage with COPS,'' Proc. 23rd ACM Symposium on Operating Systems Principles (SOSP 11), Cascais, Portugal, October 2011.
\bibitem{b2} P. Mahajan, L. Alvisi, and M. Dahlin. ``Consistency, availability, and convergence,'' Technical Report TR-11-22, Univ. Texas at Austin, Dept. Comp. Sci., 2011.
\bibitem{b3} The Linux Foundation, ``Open Glossary of Edge Computing,'' [Online]. Available: \url{https://github.com/State-of-the-Edge/glossary}
\bibitem{b4} K. Kingsbury, ``Consistency Models,'' [Online]. Available: \url{https://jepsen.io/consistency}
\bibitem{b5} M. Bayer, ``Global Galera Database,'' [Online]. Available: \url{https://review.openstack.org/600555}
\bibitem{b6} M. M. Elbushra, J. Lindstrom, ``Causal Consistent Databases'', Open Journal of Databases (OJDB), Volume 2, Issue 1, 2015.
\end{thebibliography}
\end{document}
