% This work is licensed under the Creative Commons Attribution 4.0 International
% License. To view a copy of this license, visit
% http://creativecommons.org/licenses/by/4.0/ or send a letter to Creative
% Commons, PO Box 1866, Mountain View, CA 94042, USA.
\documentclass[conference]{IEEEtran}
\IEEEoverridecommandlockouts
% That is a public draft of a posistion paper
\usepackage{cite}
\usepackage{amsmath,amssymb,amsfonts}
\usepackage{algorithmic}
\usepackage{graphicx}
\usepackage{textcomp}
\usepackage{xcolor}
\usepackage{url}
\usepackage[linguistics]{forest}
\def\BibTeX{{\rm B\kern-.05em{\sc i\kern-.025em b}\kern-.08em
    T\kern-.1667em\lower.7ex\hbox{E}\kern-.125emX}}
\begin{document}

\title{DRAFT Position Paper: Edge Clouds Multiple Control Planes Data Replication
Challenges\\
}

\author{\IEEEauthorblockN{hidden for double-blind review purposes}
\IEEEauthorblockA{\textit{} \\
\textit{}\\
\\
}
}

\maketitle

\begin{abstract}
Fog computing is an emerging paradigm aiming at bringing cloud functions closer
to the end users and data sources. Traditional DC-centric DevOps paradigms
established for cloud computing should also span to Edge clouds, which expects
them being always manageable and available for monitoring/alerting, scaling,
upgrading and applying security fixes. That is a challenge as classic methods
do not fit edge cases. Data synchronization is also a common issue for
IaaS/PaaS platforms and Mobile Edge Computing environments hosted there. In
this paper, we aim at initiating discussions around that challenge. We define
operational invariants for Edge clouds based on the Always Available autonomy
requirement and related state-of-art work. We evaluate global/replicated causal
consistent data stores and middleware as possible match for meeting those
invariants. We point out a great opportunity for designing causal consistent
systems for edge cases as a common solution for cloud infrastructure owners,
DevOps/SRE and MEC applications. Finally, we bring vision of
Replication-as-a-Service, unified design approach to benefit applications, like
NFV/StatelessNF, and cloud middleware/APIs. Having RaaS implemented as
vendors-agnostic commodity software/storage systems that provide
interoperability over hybrid clouds and multiple service providers is the
ultimate mission for future work.
\end{abstract}

\begin{IEEEkeywords}
Open source software, Edge computing, Network function virtualization,
Distributed computing, System availability, Design
\end{IEEEkeywords}

\section{Introduction}
Hybrid and multi-cloud environments is inevitable future of cloud computing.
Interconnected private and public clouds, optionally hosting
Platform-as-a-Service (PaaS) solutions on top, like OpenShift/Kubernetes, with
its workloads pushed closer to end users, unlock great potential for Mobile
Edge Computing (MEC) and massively distributed scenarios. In fog environments,
expectations for management, control and operational capabilities are pretty
same as to the traditional cloud environments, while DC-centric approaches do
not work over wide area networks (WANs) and multiple autonomous control planes.
Whereby we can have state of applications or control/management (C/M) planes
synchronized eventually and partially, which requires no strong consistency and
no global view for most of the cases and most of the time. That imposes
specific availability requirements for distributed data stores, or data
replication middleware to keep operating as an autonomous and centralized
system at once. Systems must retain its availability while being offline and
recover its state fast upon a crash. All that requires smart, WAN-optimized
state transfers and resolving of possible conflicts caused by concurrent
updates or applying locally cached operations after a network drop out ends. We
aim at initiating debates on these challenges through numerous communities,
foundations and project groups dedicated to building solutions for MEC, Network
Function Virtualization (NFV) and other edge computing cases that involve
multiple control planes and multi-site or multi-cloud operations.

\section{Background Concepts}
Aside of the established terms\cite{b3}, we define a few more for the data
processing and operational aspects:

\textbf{Deployment Data}: data that represents the configuration of
\textit{cloudlets}\cite{b3}, like API endpoints URI, or numbers of deployed
\textit{edge nodes}\cite{b3} in \textit{edge clouds}\cite{b3}. That data may
represent either the most recent state of a deployment, or arbitrary data
chunks/operations required to alter that state. When there is unresolved data
merging conflicts or queued operations pending for future execution, the most
recent state becomes the \textit{best known} one.

\textbf{Cloud Data}: similarly to deployment data, represents the most recent
or the best known internal and publicly visible state of cloudlets, like cloud
users or virtual routers. Cloud data also includes logs, performance and usage
statistics, state of message queues and the contents of databases. It may as
well represent either data chunks or operations targeted for some state
$\mathrm{S}$ transitioning into a new state $\mathrm{S'}$.

\textbf{Control Plane}: corresponds to any operations performed via cloudlets
API endpoints or command line tooling. E.g. starting a VM instance or a
Kubernetes pod. Such operations are typically initiated by cloud applications,
tenants or operators.
\begin{figure}[htbp]
\caption{Example Replication Topology}
\begin{forest}
  [$\mathrm{EAL1}$
    [\textit{access edge layer}\cite{b3}
     [\textit{infrastructure edge}\cite{b3}
       [\textit{device edge}\cite{b3}]
     ]
    ]
    [$\mathrm{EAL2}$
      [$\mathrm{EAL3}$
        [$\mathrm{cloudlet1}$
        ]
      ]
      [$\mathrm{EAL4}$
        [$\mathrm{cloudlet2}$
          [$\mathrm{EAL5}$
            [$\mathrm{cloudlet3}$
          ]
          [$\mathrm{cloudlet4}$
          ]
        ]
      ]
    ]
  ]
]
\label{fig}
\end{forest}
\end{figure}


\textbf{Replication Topology}: represents a graph for allowed state
replication flows and targets for operations on interconnected cloudlets,
including \textit{edge aggregation layers}\cite{b3}. For hierarchical
topologies, there would be a limitation for horizontally interconnected
cloudlets cannot be replicating its state nor targeting operations to each
other. This corresponds to the acyclic graph (tree) topology. There may be
also P2P topologies\cite{b9} or mixed cases.

For the example tree (Fig.~\ref{fig}), the $\mathrm{infrastructure\ edge}$ can
replicate data, like contributing its local stats into the global view of
hardware and performance metrics, to the edge aggregation layer
$\mathrm{EAL1}$ omitting the $\mathrm{access\ edge\ layer}$. Meanwhile the
latter can be pushing something, like deployment data changes, to the
$\mathrm{device\ edge}$ and $\mathrm{infrastructure\ edge}$. On the right side
of the graph, $\mathrm{cloudlet3}$ and $\mathrm{cloudlet4}$ cannot replicate to
each other, but can do to the edge aggregation layer $\mathrm{EAL1}$ either
directly or consequently via $\mathrm{EAL5}$, $\mathrm{cloudlet2}$,
$\mathrm{EAL4}$ and $\mathrm{EAL2}$, which is totally the replication
implementation specific. We also say that $\mathrm{EAL2}$ and $\mathrm{EAL1}$
are hierarchically upper associated with $\mathrm{EAL4}$ and $\mathrm{EAL5}$,
or just $\mathrm{EAL1}$ is an \textit{upper layer} for $\mathrm{EAL2}$, when we
mean that the latter is a potential subordinate of the former. Finally, we can
say that $\mathrm{EAL4}$ controls/manages $\mathrm{cloudlet2}$ and anything
sitting down of it.

That is, a C/M-subordinate hierarchical association only defines influence
domains and does not imply state replication as a mandatory thing nor imposes
bidirectional data synchronization, but rather provides an opportunity of
one-way or bidirectional state replications and regulates a possibility of
issuing C/M operations. State replication may be performed from subordinates to
its upper associates, think of sending reports. Or vice versa, think of sending
directives to subordinates. While C/M operations may only flow up-down, think
of planning future work for subordinates.

\textbf{Management Plane}: corresponds to administrative actions performed via
configuration and lifecycle management systems. Such operations are typically
targeted for cloudlets, like edge nodes, \textit{edge datacenters}\cite{b3},
or edge clouds. E.g. upgrading or reconfiguring cloudlets in a \textit{virtual
datacenter}\cite{b3}, or scaling up edge nodes. And typically initiated by
cloud infrastructure owners. For some cases, like Baremetal-as-a-Service,
tenants may as well initiate actions executed via the management plane.
Collecting logs, performance and usage statistics for monitoring and alerting
systems also represents the management plane operations, although it operates
with the cloud data.

\textbf{Causal Consistency}\cite{b6} ensures ordering of relative operations
only, where all related writes can be seen in the same order by all
processes/clients connected to all (or at least the same) server(s).
Dependencies are tracked when changing versions of objects, like key-value
pairs, and checked before making it visible elsewhere. This provides
non-blocking operations and overcomes communication latency - the primary limit
of strong consistency, where operations may block on waiting for the global
computation results. Autonomous cloudlets and edge-native applications may be
relying on such non-blocking operations.

\textbf{Data Replication Conflict}: two operations on the same target are in
conflict if they are not causal related. Such operations require further
reconciliation, i.e. resolving the conflict. Conflict-free Replicated Data
Types (CRDTs) and DRAM stores used all the way, allow minimizing overall
conflict handling work in end systems.

\textbf{Always Available (AA)}: the operational mode of the C/M planes or
applications that corresponds to the strongest (i.e. causal) consistency
possible without giving up the low-latency or non-blocking operations. That
is like \textit{sticky available causal consistency}\cite{b4} model and
\textit{Real-Time Causal}(RTC)\cite{b2}, where:
\begin{itemize}
  \item the stickiness property applies so long as clients only talk to the
    same of total $\mathrm{N}$ healthy servers, instead of switching to new
    ones.
  \item the real-time constraint requires the system time synchronized across
    cloudlets. That is a mandatory constraint for RTC, which is proved to be
    the strongest among other causal consistent data models.
  \item \textit{one-way convergence}\cite{b2} is a mandatory constraint for
    RTC.
\end{itemize}

\section{Analysis and Discussion}
\subsection{Always Available Autonomy Requirements}
We define AA autonomy for all cloudlets, including edge aggregation layers, as
the following strict requirements:

\subsubsection{Maintain multiple C/M planes}
Autonomous cloudlets may only operate as AA system when having multiple
C/M planes. F.e, when an offline cloudlet cannot start virtual machines
or containers, that violates the autonomy requirement.

\subsubsection{Provide non-blocking inter-cloudlet operations}
Operations issued by edge aggregation layers are never blocked. Depending on
the cloudlets' aliveness state shown in the global view of a particular
aggregation layer, operations may be queued for future processing but must
return futures for its state tracking. Internally, cloudlets may keep its
quorum requirements and restrict read/write operations, if needed.

\subsubsection{Provide autonomous C/M plane, whenever possible}
For offline/partitioned autonomous cloudlets, operations can be executed via
local C/M plane, if it exists and is healthy. Further steps for synchronizing
that local state with the aggregation layers or neighbor peers depend on each
particular replication topology, like either that is one-way convergent or
bidirectional, low-level data or API-level state exchange.

\subsubsection{Support long-term or permanently offline cloudlets}
Aggregation edge layers allow running an arbitrary subset of its
managed/controlled cloudlets fully autonomous for indefinitely long but keeping
the failure domains as small as possible. For causal consistent low-level data
stores this involves sloppy quorums and hinted handoffs\cite{b17}. While the
level higher, queued API operations may be served in a distributed fashion
and/or given expiration timers to alleviate the increased RAM pressure.

\subsubsection{Queue operations for the delayed replication}
Queued operations will be causally ordered and eventually replayed when the
uplink/peer connectivity restored, or expired and dropped otherwise. That poses
the \textit{delayed replication} principle. Note that we induced ordering
constraint for only related operations, like creating a VM and snapshotting it.
Global unique IDs for per-key causality may be a good fit here.

\subsubsection{Provide a global view for cloudlets aliveness state}
Global view into cloudlets' health, performance and monitoring metrics is
periodically presented for at least one of upper aggregation edge layers (or
neighbor peers). For example, with the state marks, like ``unknown'',
``autonomous'', ``synchronizing'', ``connected'', ``failed'', ``disconnected''
or ``fenced''. That poses the principle of \textit{aliveness of the cloudlets'
C/M planes}. This case may not necessary involve bidirectional data
replication and serve a good example of one-way convergent data streams.

\subsection{Operational Invariants}
To be always available, C/M planes of cloudlets must meet the following
operational capabilities (invariants):

\subsubsection{Keep C/M planes AA at best effort}
Generic API/CLI or configuration management operations, like software updates,
can always be processed by cloudlets locally and by its associated peers or
hierarchical upper layers as well. The same queuing and delayed replication
considerations as above apply here.

\subsubsection{Provide an extended global view for cloudlets}
Additionally to the aforementioned basic cloudlets' aliveness state marks,
there is extended periodic data collected and pushed for the upper edge
aggregation layers or peers. At least for the adjacent (next-hop) cloudlets,
that covers such metrics as hardware status and power management, logging,
monitoring and alerting events, performance and metering statistics.

\subsubsection{Leverage Replication-as-a-Service solutions for storing state}
Replication-as-a-Service (RaaS) allows detaching state from C/M planes and
edge-native applications, like StatelessNF\cite{b16}. That turns the such into
generic stateless workloads easy to manage/migrate across different hybrid
clouds and/or orchestration engines. While all the complexity of maintaining
those to be AA and synchronized shifts to that particular RaaS solution.

\subsection{State Replication Consistency Requirements and Recommendations}
As it follows from the defined AA autonomy requirements and operational
invariants, there are associated state replication requirements\footnote{when
we refer to \textit{data} or \textit{state}, we do not differentiate either
that is deployment/cloud data, internal state of an NFV application or cloud
API operations. That poses the \textbf{unified approach principle.}}:

\subsubsection{Incorporate convergent conflict handling}
Conflicting operations can only be maintained as causal related after
reconciled, either internally for data replicas or with application-specific
functions upon accessing such conflicting data. Last-writer-wins (LWW)
systems\cite{b1}\cite{b13} may be best for reconciling operations issued by
multiple control planes. While ``return them all''\cite{b17} strategy better
fits the configuration management tasks performed on autonomous cloudlets.
Whereby conflicts may have to be reconciled differently and depend on either
the configuration was applied locally and is replicated back to the upper
aggregation layers, or pushed down after the autonomous cloudlet recovers its
uplink connection to the management plane. Or when an NVF application performs
a handover for user mobility cases in 5G networks. The source of truth then may
be based on the distance/location instead of timestamps.

\subsubsection{Prefer per-key causal and one-way convergent replication topologies}
Inter-cloudlets data replication is costly and should be as minimal as possible.
When it is inevitable, prefer one-way convergence\cite{b2} over bidirectional
data replication. This greatly simplifies end systems implementation. And
per-key causality is well covered in the existing data store systems (see the
related work section).

\subsection{Data Replication Challenges}
All that brings us to challenges that need to be addressed for multiple C/M
planes of edge clouds:
\begin{itemize}
  \item identifying types of C/M operations and replicated
    data, when grouping those into particular replication topologies. Such
    groups may be identified by multiple metrics, like communication latency,
    tolerated duration of network partitions for offline cloudlets, one- or
    two-way convergence based replication of either low-level data or
    operations at the higher API-to-API levels.
  \item the unified replication topology should not bring excessive operational
    overhead, like maintaining all of the identified types of operations
    simultaneously for the end system, and require not too much of human care.
  \item picking the right replication topologies for each of the involved
    system components, like an identity provider or images streaming services.
    F.e., an efficient implementation of caching of VMs/containers images may
    require maintaining a dynamic peer-to-peer mesh topology replicated across
    adjacent cloudlets.
  \item programming CRDTs for the conflicts-free systems or ``smart'' conflicts
    resolvers based on the picked replication topologies.
  \item keeping the state replication topologies always efficient and adjusting
    itself dynamically. E.g., when executing a handover for a mobile subscriber
    in a 5G network, an orchestrator must identify the adjacent endpoints based
    on the subscriber location and define the numbers of replicas to place
    there. Or distributing the queued control plane operations targeted for the
    associated offline cloudlets might require more of the nested edge
    aggregation layers or peers to be added dynamically or statically, when the
    load exceeds hardware capabilities of a particular aggregation site.
  \item programming middleware that abides the unified approach and fits the
    targeted RaaS cases for IaaS/Paas control and management planes as well as
    the edge-native, like NFV, workloads. For most of the cases, one size
    does not fit all, so future RaaS solutions should be tightly coupled with
    its suggested replication topologies.
\end{itemize}

\begin{itemize}
  \item \textbf{Open questions}: what Edge computing cases involving autonomous
    multiple control planes can be expressed via one-way convergent causal
    per-key consistency?
\item \textbf{Open questions}: for the remaining two-way convergent cases,
    may an API-to-API state synchronization replace the low-level data
    replication?
\end{itemize}

\subsection{Vision of a Unified State Replication As a Service}
The vision of the unified architecture for data replication tooling is solving
common problems for C/M planes of IaaS/PaaS, end users and edge-native
applications. F.e., generic version control systems are well suited for code
and might fit all cases for configuration data replication and conflicts
resolving. But that fits only specific needs of management planes and cannot be
a unified RaaS solution as we see it.

For OpenStack, rearchitecting its relational data models for causal consistency
is challenging, whenever it is reimplemented for a key-value store (KVS) or
added new middleware for replicating data across multiple clusters. But
ultimately that enables the most seamless integration of OpenStack and
Kubernetes platforms into a powerful RaaS solution with a potential of
targeting also MEC environments and Telco NFV cases.

Client libraries implementing such RaaS solutions may provide that unification
layer via replicating data across different KVS/DB backends and/or replicating
operations at API-to-API levels. While the cloudlets' local data stores may
share nothing and provide \textit{unavailable}\cite{b4} strong consistency for
internal/static state.

Global geo-distributed causal consistent data stores may become another viable
alternative for unified RaaS solution. Although that would be less flexible
than middleware supporting multiple data backends.

\section{Related Work}
For OpenStack, there is StarlingX\cite{b24} that provides a framework for
low-level data synchronization over relational database clusters. And there is
Juice\cite{b21} evaluation of CockroachDB\cite{b22} replacing the traditional
relational database backend.

Global Causal Consistent Databases\cite{b6} work presents potential
applications and databases that could use the causal consistency and shows
possible methods to implement that model. It also compares
\textit{serializability}\cite{b4}, eventual and causal consistency using a
running example. The work concludes that there is no commercial or mature
systems implementing the causal consistency for a global data store.

There are open-source systems that implement Dynamo\cite{b17} per-key causal
consistency: Riak\cite{b19} and LinkedIn's Voldemort\cite{b20}. Dynamo
provides an ``always writable'' data store where no updates are rejected due
to failures or concurrent writes and is conceptually close to a ``zero-hop''
distributed hash table (DHT) system. It is DC-centric and assumes all cluster
nodes are trusted. To overcome such limitations of a single administrative
domain, there is an additional middleware layer may be needed to perform data
replication across multiple isolated clusters. Like Indigo middleware\cite{b10}
that gives strong fundamentals on creating application-centric programming
methodologies that leverage invariant-repair/violation-avoidance techniques and
rely on CRDTs. Ultimately, that should help programmers to enforce Explicit
Consistency by extending existing applications logic and building middleware
libraries that operate on top of the underlying causal consistent data stores.

Eiger\cite{b1} presents geo-replicated data store where causal consistency is
tracked per operations instead of per-key. In indroduces read-only and
write-only transactions for keys spread across many servers. Such transactions
are atomic within a datacenter, but not globally, which guarantees low-latency
and never-blocking operations. Eiger supports Cassandra's\cite{b25} rich data
model (richer than simple key-value pairs) pioneered by Google's Bigtable\cite{b26},
which was designed to scale to petabytes of data across thousands of commodity
servers. There is also a research fork of Cassandra implementing Eiger\cite{b23}.

Walter\cite{b11} KVS uses a set-like CRDTs. It provides a strong consistency
guarantee within a site and causal ordering across sites. It introduces
Parallel Snapshot Isolation (PSI) property that extends snapshot isolation by
allowing different sites to diverge with different commit orderings. This is
also known as sticky available causal consistency model. Walter performs
asynchronous replication across multiple sites and supports partial replication
for multi-cluster scenarios to address scalability bottlenecks.

SwiftCloud distributed object database\cite{b12} brings geo-replication to the
client machines instead. It supports interactive transactions that span
multiple CRDT types. To its authors' knowledge, asynchronous replication
systems ensure fault-tolerant causal consistency only within the boundaries of
the DC, while SwiftCloud guarantees convergent causal consistency all the way
to resource-poor end clients. SwiftCloud also proposes a novel client-assisted
failover mechanism that trades latency by a small increase in staleness of
data.

Bolt-on\cite{b13} shim takes another approach and allows to leverage existing
production-ready eventually consistent data stores virtually upgrading it to
provide convergent causal consistency.

STACK Research Group\cite{b8} provides a list of the features required to
operate and use edge computing resources. The listed requirements are
complementary to this work and represent the operational invariants approached
from OpenStack developers and operators (DevOps) angle, the view point that
also covers interoperability between multiple operators. The latter is an
important requirement for hybrid clouds and NFV Edge cases, like Virtual
Customer Premises Equipment (vCPE).

In Mobile Edge Computing environments, there is also high demand for novel
lightweight data replication and applications live migration solutions. Those
must perform well over WAN and not being DC-centric. Proactive data replication
techniques to cope with user mobility considered in the related work\cite{b14}.
It poses challenges of efficient scheduling of data replicas over the edge
nodes. According to ETSI reference architecture, MEC Orchestrators are
responsible for solving these problems via user mobility prediction, while
virtualization infrastructure management (VIM) should implement the proposed
procedures of proactive migration. It is notable that container‐based VIMs are
considered the most promising solutions for MEC environments, mostly due to
reduced footprint of containers. Virtualized 5G in Evolved Packet Core (vEPC)
Architecture\cite{b15} serves another good example of high demand emerging
from the world of Telcos. The work describes a state sharing mechanism across
different datacenters that leverages Edge Synchronization Protocol(ESP) and
Abstract Syntax Notation.1 (ASN.1). None of these two works mention
causal consistency but it reads between the lines as the such, that answers the
questions, like which state portions should be replicated, to which cloudlets
and under which conditions. E.g. predicted users' trajectories or operations
involving particular mobile subscribers may help to establish causal relations
and produce ultimate replication decisions at the application levels.
StatelessNF\cite{b16} rearchitects NFV applications to decouple its internal
dynamic state, like connections tracked by firewalls, into a low-latency DRAM
storage tier of RAMCloud\cite{b18}. It provides 100-1000x lower latency than
disk-based systems and 100-1000x greater throughput, which greatly reduces
possibility of concurrent updates causing non-mergeable data conflicts. While
this benefits NFV cases a lot, RAMCloud per se can not meet the defined autonomy
requirements for C/M planes, where concurrent data updates and cross-datacenter
replication is inevitable. In such setups, RAMCloud as a potential RaaS
solution has no more its low-latency advantages. But other DRAM stores might
still be helpful as a local caching layers.

A. Lebre et al.\cite{b9} bring vision of a Peer-to-Peer (P2P) OpenStack IaaS
Manager, which is opposed to hierarchical distributed topologies. It is stated
that P2P saves maintenance costs associated with hierarchies, while the latter
also require complex operations in case of failures. The unified IaaS Manager
is positioned as an alternative to stretched control plane and federated
clouds. It emphasizes on challenges of data locality, efficient cloud storage,
interoperability peering agreements, autonomous lifecycle tooling and new
edge-native cloud APIs as paramount requirements. Such a view-point naturally
complements the vision of unified management plane as we introduced it for this
paper.

The edge-cloud native APIs may be also designed with the approaches that
Indigo\cite{b10} or RainbowFS\cite{b7} takes. The latter work focuses on
Just-Right (modular) Consistency and flexible service level agreements, like
latency requirements, enforced data locality and smart storage placement
strategies unique to Fog/Edge platforms and dictated by applications designed
for it on case-by-case basis. ``Whereby an application pays only the
consistency price that it strictly requires''\cite{b7}.

\begin{itemize}
  \item \textbf{Open questions}: as RaaS targeted for OpenStack, Kubernetes and
    NFV applications, what mature (or in active development) open-source causal
    consistent data store fits the identified edge computing cases, given the
    defined operational invariants?  E.g.\cite{b12}\cite{b19}\cite{b20} and
    \cite{b23} are open-source projects available under Apache 2.0 license and
    might be a good start for building such RaaS system. Augmenting it with
    local in-memory caching might end up in a powerful option for NFV as well.
\end{itemize}

\section{Conclusion}
We defined autonomy requirements for multiple autonomous C/M planes of Fog
environments and imposed operational invariants. That brought us to the causal
consistency requirement for inter-cloudlets multi-DC state replication. We
highlighted challenges of building replication topologies and propposed design
principles, like low-level data replication and API-level queuing with delayed
replication, aliveness of the cloudlets' C/M planes. We brought vision of a
unified approach for building ultimate RaaS solutions for Fog/MEC environments.
Such solutions may be based on theoretical/research global causal consistent
systems, like \cite{b11}\cite{b12}. Or middleware shims providing multi-site
replication on top of traditional DC-centric data stores, like strongly
consistent\cite{b22}, eventual/per-key causal consistent\cite{b19}\cite{b20},
or stores implemented from theoretical/research systems
\cite{b1}\cite{b2}\cite{b7}\cite{b10}\cite{b13}\cite{b23}.

Alternatively, there may be middleware that queues and replicates C/M plane
operations at an API-to-API high-level. This might better fit the
interoperability requirements of hybrid OpenStack/Kubernetes (plus public
clouds) architectures and simplify end-to-end implementation.

We want to emphasize on the unification principle as the great opportunity for
developers to bring the best of IaaS and PaaS worlds for end users whom such
data replication tooling might benefit as a service, i.e. DevOps and tenants
workloads, like MEC environments and NFV. It is also an opportunity for Telco
side to leverage the commodity replication services of infrastructure layers
for the hosted virtual network functions (VNFs). The unification principle also
applies to any fog-based system and is not limited to OpenStack and Kubernetes
platforms or NFV cases. We only focus on those as a possible good start.

As of today, OpenStack and Kubernetes cloud platforms rely on DC-centric data
stores that provide only strong consistency guarantees and do not scale into
fog environments with multiple clusters and latency-sensitive workloads.
Neither there is support for causal consistent data backends, nor middleware
capable driving of replication of casual related state. We see future work in
integrating OpenStack and Kubernetes as a hybrid edge computing platform that
leverages a common RaaS solution. Whereby its supported data backend may be
DHT-like open-source derivatives of\cite{b17} KVS or Bigtable-like systems for
richer data models. With a caveat for novel StatelessNF applications's dynamic
state must be DRAM based, like \cite{b18}, or assisted with local in-memory
caching backends. That RaaS should also include middleware layers that provide
programming interfaces for proxying of API operations (i.e. queuing/dequeuing
with causal ordering preserved) and atomic updates/transactions\cite{b1}. So
end clients can access it in a causally consistent and always available
fashion. Finally, edge-native applications and cloud infrastructure components
(the end clients) on case-by-case basis need to implement its conflicts-free
CRDT types and/or semantic reconciliation protocols. Whenever applicable, an
API-to-API synchronization may be prefered over low-level data operations and
one-way convergent causality over bidirectional state replication.

To draw the line for where/when future work starts, we should remember
that ``retain workloads operational as the best effort'' principle provides a
good start for minimum viable products (MVP), like Distributed Compute Node
(DCN) scenario. It is also known to be highly wanted by Telco segment for its
next-generation Virtual Radio Access cellular Networks (vRAN).

A centralized C/M plane enables early implementations for DCN as it has no
autonomy requirements. While post-MVP phases should be evaluated for the most
advanced cases, like MEC vEPC, IoT, AI, autonomous aerial drones, big data
processing at the edge networks etc. That is where the data replication
challenges to fully arise for expensive handover procedures, mobile networks'
subscribers state transfers and just generic Day-2 operations of multiple
C/M planes of Edge clouds.

\section*{Acknowledgment}
Thanks to all who reviewed this paper, including but not limited to:
Reedip Banerjee, from Red Hat; Flavia Delicato and Paulo Pires, from Federal
University of Rio de Janeiro. Special thanks to all of the participants of
OpenStack Summit at Berlin, 2018.

\begin{thebibliography}{00}
\bibitem{b1} W. Lloyd, M. J. Freedman, M. Kaminsky, D. G. Andersen, ``Stronger Semantics for Low-Latency Geo-Replicated Storage'', NSDI'13, 2013.
\bibitem{b2} P. Mahajan, L. Alvisi, M. Dahlin. ``Consistency, availability, and convergence,'' Technical Report TR-11-22, Univ. Texas at Austin, Dept. Comp. Sci., 2011.
\newpage
\bibitem{b3} The Linux Foundation, ``Open Glossary of Edge Computing,'' [Online]. \newline Available: \url{https://github.com/State-of-the-Edge/glossary}.
\bibitem{b4} K. Kingsbury, ``Consistency Models,'' [Online]. Available: \newline \url{https://jepsen.io/consistency}.
\bibitem{b5} M. Bayer, ``Global Galera Database,'' [Online]. Available: \newline \url{https://review.openstack.org/600555}.
\bibitem{b6} M. M. Elbushra, J. Lindstrom, ``Causal Consistent Databases'', Open Journal of Databases (OJDB), Volume 2, Issue 1, 2015.
\bibitem{b7} PRCE (Projet de recherche collaborative — Entreprises), ``RainbowFS: Modular Consistency and Co-designed Massive File'' [Online]. Available: \url{http://rainbowfs.lip6.fr/data/RainbowFS-2016-04-12.pdf}.
\bibitem{b8} R. A. Cherrueau, A. Lebre, D. Pertin, F. Wuhib, J. M. Soares, ``Edge Computing Resource Management System: a Critical Building Block! Initiating the debate via OpenStack'', USENIX HotEdge’18 Workshop, 2018.
\bibitem{b9} A. Lebre, J. Pastor, A. Simonet, F. Desprez, ``Revising OpenStack to Operate Fog/Edge Computing Infrastructures'', IC2E, 2017.
\bibitem{b10} V. Balegas, N. Preguiça, R. Rodrigues, S. Duarte, C. Ferreira, M. Najafzadeh, M. Shapiro, ``Putting Consistency back into Eventual Consistency'', EuroSys, 2015.
\bibitem{b11} Y. Sovran, R. Power, M. Aguilera, J. Li, ``Transactional Storage for Geo-replicated Systems'', SOSP, 385-400, 2011.
\bibitem{b12} M. Zawirski, A. Bieniusa, V. Balegas, S. Duarte, C. Baquero, M. Shapiro, and N. Preguica, ``SwiftCloud: Fault-Tolerant Geo-Replication Integrated all the Way to the Client Machine'', Research Report RR-8347, INRIA, 2013.
\bibitem{b13} P. Bailis, A. Ghodsi, J. M. Hellerstein, I. Stoica, ``Bolt-on Causal Consistency'', SIGMOD, 761–772, 2013.
\bibitem{b14} I. Farris, T. Taleb, H. Flinck, A. Iera, ``Providing ultra‐short latency to user‐centric 5G applications at the mobile network edge'', Transactions on Emerging Telecommunications Technologies, Vol. 29, No. 4, e3169, 2018.
\bibitem{b15} E. Cau, M. Corici, P. Bellavista, L. Foschini, G. Carella, A. Edmonds, T. Bohnert, ``Efficient Exploitation of Mobile Edge Computing for Virtualized 5G in EPC Architectures'', MobileCloud, 100-109, 2016.
\bibitem{b16} M. Kablan, A. Alsudais, E. Keller, Franck Le,``Stateless Network Functions: Breaking the Tight Coupling of State and Processing'', NSDI, 97-112, 2017.
\bibitem{b17} G. DeCandia, D. Hastorun, M. Jampani, G. Kakulapati, A. Lakshman, A. Pilchin, S. Sivasubramanian, P. Vosshall, W. Vogels, ``Dynamo: Amazon's Highly Available Key-Value Store'', SOSP 2007: 205-220, 2007.
\bibitem{b18} RAMCloud: \url{https://ramcloud.stanford.edu}.
\bibitem{b19} Riak: \url{https://docs.riak.com/riak/kv/latest/learn/dynamo/index.html}.
\bibitem{b20} Voldemort: \url{https://github.com/voldemort/voldemort}.
\bibitem{b21} Juice: \url{https://github.com/BeyondTheClouds/juice}.
\bibitem{b22} CockroachDB: \url{https://www.cockroachlabs.com/docs}.
\bibitem{b23} Cassandra-Eiger: \url{https://github.com/wlloyd/eiger}.
\bibitem{b24} StarlingX: \url{https://www.starlingx.io/}.
\bibitem{b25} A. Lakshman, P. Malik. ``Cassandra – a decentralized structured storage system''. LADIS, 2009.
\bibitem{b26} F. Chang, J. Dean, S. Ghemawat, W. C. Hsieh, D. A. Wallach, M. Burrows, T. Chandra, A. Fikes, R. E. Gruber, ``Bigtable: A distributed storage system for structured data'', ACM TOCS, 26(2), 2008.
\end{thebibliography}
\end{document}
